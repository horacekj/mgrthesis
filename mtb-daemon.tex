\section{MTB daemon}

\gls{mtb} daemon je první ze dvou počítačových aplikací, která v~rámci této
práce vznikla. Jejím hlavním úkolem je připojit se k~\gls{mtbusb} modulu
a umožnit jeho ovládání více různým aplikacím. K~tomuto účelu aplikace vystavuje
TCP server, ke kterému se řídicí programy mohou připojit.

\subsection{Protokol TCP serveru} \label{sec:daemon:proto}

TCP server by měl umožňovat oboustrannou komunikaci – jak od klienta k~serveru
(\textit{požadavek, odpověď}, tak od serveru ke klientovi (\textit{asynchronní
události}). Například protokol \textit{http} v~kombinaci s~\textit{REST API}
tedy není možné použít. Z~běžné používaných technologií byly zvažovány
\textit{websockets}, tyto ale nebyly využity kvůli zbytečně složité inicializaci,
která probíhá přes protokol \textit{http}.

Autor zvolil snad nejjednodušší možné řešení – TCP spojení se udržuje trvale
otevřené, posílají se textová data, každý řádek je jedna zpráva. Textová data
jsou ve formátu \textit{json}. Textový protokol byl zvolen, aby bylo možné
provádět snadnou diagnostiku. Očekává že, že řídicí aplikace se bude připojovat
převážně ze stejného počítače, na kterém běží \gls{mtb} daemon, šetřit kapacitu
linky binárním protokolem tedy nemá příliš velký smysl. Formát \textit{json}
byl zvolen proto, že je dnes moderním standardem formátu výměny dat mezi
aplikacemi. Snad všechny majoritní programování jazyky obsahují ve svých
standardních knihovnách podporu pro práci s~\textit{json}.

Zprávy se posílají ve standardním kódování \textit{UTF-8}. Uveďme několik
příkladů zpráv (pro přehlednost je zpráva rozdělena na více řádků, ve skutečnosti
se posílá na jednom řádku):

\begin{verbatim}
{
    "command": "mtbusb",
    "type": "request",
    "id": 42
}
\end{verbatim}

\begin{verbatim}
{
    "command": "mtbusb",
    "type": "response",
    "id": 42,
    "status": "ok",
    "mtbusb": {
        "connected": true,
        "type": 1,
        "speed": 115200,
        "firmware_version": "1.0",
        "protocol_version": "1.0",
        "active_modules": [1, 5, 2, 121]
    }
}
\end{verbatim}

Zprávy jsou tří typů \texttt{type}:

\begin{compactenum}
\item \texttt{request} (zpráva od klienta serveru),
\item \texttt{response} (odpověď na \texttt{request}),
\item \texttt{event} (asynchronní událost od serveru ke klientovi).
\end{compactenum}

Aby klient věděl, která odpověď patří kterému požadavku (klient může posílat
více požadavků rychle za sebou), lze u~každého požadavku uvést \texttt{id}
(číslo). Příslušná odpověď pak obsahuje stejné \texttt{id}.

Klient se může zaregistrovat ke konkrétním modulům zprávou
\texttt{module\_subscribe} (a odregistrovat zprávou
\texttt{module\_ubsubscribe}). Pokud je klient zaregistovaný k~nějakému modulu,
chodí mu asynchronní události o~změně stavu modulu (např. změna stavu vstupu).
Klienti si tak mohou volit, o~které moduly se starají.

Kompletní specifikace protokolu je k~dispozici na
\url{https://github.com/kmzbrnoI/mtb-daemon/tree/master/tcp-protocol}.

\subsection{Volba nástrojů} \label{sec:daemon:tools}

\gls{mtb} daemon je klíčová aplikace infrastruktury řízení kolejiště. Proto se
autor rozhodl ji programovat v~programovacím jazyce se silným statickým typovým
systémem. Dalším klíčovým faktorem zvolit vhodné knihovny pro přístup
k~sériovému portu, TCP serveru a zpracovávání \textit{json}. V~neposlední řadě
pak samozřejmě zkušenosti autora.

Jako programovací jazyk bylo zvoleno \texttt{C++} (\texttt{C++17})
s~frameworkem \texttt{Qt}, který nabízí přímočaré knihovny pro komunikaci se
sériovým portem a pro síťovou komunikaci. U~knihoven zejména není třeba
složitěji řešit vlákna. \texttt{C++} a \texttt{Qt} umožňují multiplatformní
řešešení aplikace – \gls{mtb} daemon lze zkompilovat a spustit jak na OS Linux,
tak OS Windows, což byl další požadavek autora.

\subsection{Implementace} \label{sec:daemon:impl}

Aplikace \gls{mtb} daemon je psána jako jednovláknová konzolová aplikace.
Při návrhu byla zvažována i možnost implementovat aplikaci jako okenní
(operátor obsluhy by přes aplikaci konfiguroval \gls{mtb} desky a měl by
přehled o~stavu systému), nakonec byla však zvolen přístup, kdy aplikace pouze
vystavuje API. Konfigurace se načítá ze souboru. V~případě zájmu o~vytvoření
grafického nástroje pro interakci se systémem \gls{mtb} bude tento systém
vytvořen jako další z~klientů aplikace \gls{mtb} daemon nebo jako nástroj pro
přímou úpravu konfiguračního souboru.

Při implementaci bylo třeba vyřešit, jak zaručit, aby sběrnici nemohli ovláadt
neautorizovaní klienti. V~současnosti je implementován pouze velice hrubý
autorizační mechanismus – server poslouchá pouze na specifických rozhraních,
typicky pouze \textit{localhost}. Očekává se, že lokální stroj je server
kolejiště, na kterém by měly běžet pouze bezpečné aplikace. V~budoucnu možná
autor přistoupí k~implementaci autentizace založené na povolení konkrétních
IP adres klientů. Očekává se, že klienti aplikace \gls{mtb} daemon jsou trvale
připojená nemigrující síťová zařízení, která řídí nějaký konkrétní systém
kolejiště, proto dle autora není nutné implementovat žádný složitější
autentizační mechanismus.

Jednotliví klienti mohou ovládat libovolné výstupy libovolných \gls{mtb} desek.
Při návrhu vznikla otázka, zda by v~konfiguraci nemělo být specifikováno,
který klient může ovládat který výstup (které desky). Po uvážení však bylo od
tohoto přístupu upuštěno, protože by vyžadovat netriviální (ručně psanou)
konfiguraci navíc a identifikaci klientů. Očekává se, že klienti připojují
se k~\gls{mtb} daemon jsou buď autonomní aplikace, nebo aplikace, které řízení
přístupu svých uživatelů implementují na své vlastní úrovni. Pokud by nastala
problematická situace, kdy 2 klienti chtějí ovládat stejný výstup, jedná se
tedy o~chybu v~klientské aplikaci nebo její konfiguraci, nikoliv nekalý záměr,
kterému by měl systém vší silnou zabránit. V~duchu tohoto tvrzení \gls{mtb} daemon
takové chování umožní a zaloguje, že 2 klienti nastavují stejný výstup. To
umožňuje operátorovi takovou situaci detekovat a konfiguraci opravit.

\gls{mtb} daemon si pamatuje, který klient nastavil který výstup a při odpojení
klienta provede reset výstupů do bezpečného stavu.

Aplikace obsahuje hierarchii tříd (ve vztahu dědičnosti) reprezentující
jednotlivé typy \gls{mtb} modulů. TCP protokol také počítá s~různými typy
modulů. Aplikace je tak připravena na přidání budoucích nových typů \gls{mtb}
modulu.

\textit{MTB daemon} je pod opensource licencí dostupný na
\url{https://github.com/kmzbrnoI/mtb-daemon}.
