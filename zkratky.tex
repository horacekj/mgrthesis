\newglossaryentry{plc} {
	name=PLC,
	description={Programmable Logic Controller, robustní zařízení průmyslové
	atumatizace}
}

\newglossaryentry{mtb} {
	name=MTB,
	description={Hardwarový systém pro řízení modelových kolejišť složený ze
	sběrnice MTBbus, MTB modulů a~MTB-USB desky}
}

\newglossaryentry{mtbbus} {
	name=MTBbus,
	description={Model Train Bus\footnote{Expanze zkratky do jejího plného
	významu nedává smysl, i tak budeme používat označení MTBbus, protože tak je
	zkratka zaužívaná.}, sběrnice určená pro řízení modelových kolejišť}
}

\newglossaryentry{dcc} {
	name=DCC,
	description={Digital Command Control, mezinárodně užívaný standardizovaný
	systém pro digitální řízení modelové železnice}
}

\newglossaryentry{kmz}{
	name={KMŽ Brno~I},
	description={Klub modelářů železnic Brno~I}
}

\newglossaryentry{nmra}{
	name={NMRA},
	description={National Model Railroad Association}
}

\newglossaryentry{mtbusb}{
	name={MTB-USB},
	description={Master modul sběrnice MTBbus, implementuje rozhraní mezi MTBbus
	a počítačem (USB)}
}

\newglossaryentry{mtbuni}{
	name={MTB-UNI},
	description={Nejrozšířenější slave modul sběrnice MTBbus, má 16~digitálních
	vstupů a~16~digitálních výstupů}
}

\newglossaryentry{ttl}{
	name={TTL},
	description={Transistor-transistor-logic, v~kontextu této práce standard
		definující jaká napěťová úroveň odpovídá jaké logické úrovni}
}

\newglossaryentry{usb}{
	name={USB},
	description={Universal Serial Bus, v současnosti nejpoužívanější sběrnice
	pro připojení periferií k počítači}
}

\newglossaryentry{cdc}{
	name={CDC},
	description={USB Comunications Device Class, třída USB protokolu implementující
		tunelování sériové linky skrze USB}
}

\newglossaryentry{dps}{
	name={dps},
	description={Deska plošnýcn spojů}
}

\newglossaryentry{scom}{
	name={S-COM},
	description={Jednoduchý jednosměrný komunikační protokol, který umožňuje
	přenášet návěsti návěstidel \cite{scom-specs}}
}

\printglossary[title=Seznam použitých zkratek]

