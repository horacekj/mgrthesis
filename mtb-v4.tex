Našim základním cílem v této kapitole bude navrhnout \gls{plc} systém, který
splňuje požadavky definované na konci kapitoly \ref{chap:nasazeni}. Při návrhu
tohoto systému vyjdeme z přirozeného požadavku na minimální změny v aktuálním
harwaru a softwaru kolejiště. Navrhneme nový systém jako iteraci současného
systému \gls{mtb}. Tento systém pojmenujeme \textit{MTB v4}.

Nový systém vyřeší všechny problémy aktuální verze systému \gls{mtb} popsané
v kapitole \ref{chap:nasazeni}. Při návrhu vezmeme v potaz aktuální technologie
v oblasti \gls{plc} obvodů a zvážíme implementace těcht technolgií. Detailně
rozebereme současnou implementace protokolu \gls{mtbbus} a u každé položky
zvolíme, jestli a proč ji chceme zachovat.

\section{Podrobná specifikace požadavků}

Abychom byli schopni navrhnout MTB v4, musíme znát přesné požadavky na takový
systém. Nyní požadavky formulujeme, přičemž budeme klást důraz na vysvětlení
změn oproti současnému systému \gls{mtb}.

\subsection{Interakce se systémem MTB}

Zaměřme se nejprve po požadavky, které na systém \gls{mtb} kladou komponenty,
které tento systém používají.

\subsubsection{Hardware v kolejišti}

K \gls{mtb} se na straně kolejiště připojují jednotlivé vstupy a výstupy
popsané v kapitole \ref{chap:existujici-reseni}. Všechny v současnosti na
kolejišti používané vstupy jsou digitálního formátu (binární vstupy), výstupy
jsou buď digitální binární nebo S-COM \ref{scom}. S-COM výstupy jsou ve
skutečnosti digitální výstupy, do kterých MTB deska moduluje S-COM signál.
S-COM signál je jednoduchý pomalý signál, který lze s přehledem vytvářet běžným
výstupním pinem procesoru nebo posuvného registru.

Oproti současnému \gls{mtb} bychom nově chtěli podporovat \textit{kmitavý
výstup} s počítačově definovou frekvencí kmitání a pevnou střídou. Jednalo by
se o kmitání v řádu jednotek \textit{Hz}, využití tohoto typu výstupu je pro
indikace v pultech a rozpojovače \ref{rozp}. I toto je však z pohledu hardwaru
jednoduchý digitální výstup, pro přidání tohoto typu výstupu je třeba upravit
pouze komunikační protokol a software a firmware modulů.

Současné moduly \textit{MTB-UNI} umožňují výstupy typu S-COM pouze na pinech
0–7, což je omezení dané malou pamětí procesoru. Toto omezení by mělo
\textit{MTB v4} relaxovat a umožnit tak, aby každý výstup mohl být v jednom
z režimů

\begin{compactenum}
\item digitální,
\item S-COM,
\item kmitavý.
\end{compactenum}

Co se týče vstupně-výstupních požadavků, \textit{MTB v4} by si vystačilo pouze
s moduly typu \textit{MTB-UNI} \ref{uni}. Na kolejišti v současnosti není žádný
prvek, který by vyžadoval například analogový vstup, analogový výstup nebo např.
\textit{PWM} výstup. Na kolejištích se hojně používají servomotory, ale vždy
pro řízení dvoupolohových prvků: výhybka, výkolejka, mechanické návěstidlo.
Všechny tyto prvky k sobě mají řídicí elektroniku, která interaguje se systémem
\gls{mtb} pouze digitálními piny. Typicky stačí 2 digitální výstupy (pro
nastavení požadované polohy) a případně 2 digitální vstupy (pro detekci koncové
polohy).

Zmiňme na tomto místě, že neimplementace PWM výstupů do \textit{MTB-UNI v4}
desky je koncepčním rozhodnutím autora této práce. Autor práce razí přístup
\uv{ať jedna věc dělá jeden úkol a dělá ho dobře}. Místo návrhu všemocného
\textit{MTB-UNI} modulu, který za pár let bude umět i uvařit kávu, se autor
rozhodl vydat cestou univerzálního kompaktního snadno sériově vyrobitelného
modulu s prostými digitálními vstupy a výstupy.

Připomeňme, že \textit{MTB-UNI} desky umožňují speciální typ vstupů – IR vstupy,
viz sekci \ref{sec:uni_ir}. Tyto vstupy vyžadují další elektroniku navíc.
V nových \textit{MTB-UNI v4} deskách bude podpora pro IR čidla zrušena ve jménu
předchozího odstavce. Bude navržena speciální deska pro buzení a vyhodnocování
stavu IR čidel, která se bude připojovat přímo k digitálním vstupům
\textit{MTB-UNI} desky. Ušetří se tak elektronika na \textit{MTB-UNI} deskách,
na kterých se IR vstupy nevyužívají, zjednoduší se návrh desky plošných spojů
a umožní se využít IR čidla i s jinými deskami, než MTB, např. pro přímou
indikaci v pultech. Jedna deska bude dělat jednu věc a bude ji dělat dobře.

Na současném kolejišti si tedy vystačím pouze s \textit{MTB-UNI} deskami, které
budou navíc ořezané o podporu IR čidel. Ačkoliv je toto tvrzení pravdivé, bylo
by neperspektivní si podporu jiných typů desek zavřít volbou nevhodného
protokolu. Nový protokol sběrnice \gls{mtbbus} by měl být navržený univerzálně
pro nejrůznější možné typu desek, které v budoucnu mohou přijít. Autor této
práce má již v záměru takové desky desky vytvořit, více se můžete dočíst
v budoucí kapitole \ref{todo}.
