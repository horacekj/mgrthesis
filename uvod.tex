\textit{Programmable Logic Controller} (\gls{plc}) je vestavěný počítač,
který zpracovává vstupy, provádí jejich vyhodnocování, komunikuje s dalšími
\gls{plc} obvody a nastavuje výstupy \cite{plc:web}. Je základní jednotkou průmyslové
automatizace. \gls{plc} obvody se používají například pro řízení procesů
výrobních linek, kde zpracovávají data z~nejrůznějších čidel a~ovládají
periferie – například motorické pohony, signalizační diody, robotické ruky,
lasery apod.

Obvody \gls{plc} jsou zcela klíčové v tom, že tvoří rozhraní mezi hardwarovými
periferiemi (například motorem) a počítačem, který zpracovává data. Na
\gls{plc} obvody budeme v této práci nahlížet jako na vstupně–výstupní moduly,
jejichž hlavním úkolem je správně zpracovávat vstupní signály a správně ovládat
výstupní periferie. Řízení samotného procesu přenecháme počítači, s kterým
\gls{plc} moduly komunikují. \footnote{Toto není jediná možná aplikace \gls{plc}
obvodů – obvod samotný může mít vlastní inteligenci, zde však tuto situaci
zkoumat nebudeme.} Pro propojení modulů se používá například \textit{Ethernet}
nebo sběrnice \textit{ModBus} \cite{modbus:web}.

Cílem této práce je navrhnout a implementovat novou verzi sběrnice pro řízení
\gls{plc} obvodů zvanou \gls{mtbbus}.

\gls{mtbbus} je sběrnice, která se v současnosti využívá pro řízení modelových
kolejišť, kde plní podobnou funkci, jako v příkladu s výrobní linkou – čte
signály z kolejiště (například polohy výhybek, obsazenost kolejových obvodů
\cite{ko:web}) a poveluje prvky v kolejišti (návěstidla, přestavníky výhybek,
přejezdy apod.). Obdobným způsobem funguje zapezpečovací zařízení na skutečné
železnici. Sběrnice \gls{mtbbus} není určená k řízení jízdy hnacích vozidel,
tuto funkci plní jiné sběrnice.

Primární využití sběrnice \gls{mtbbus} definuje požadavky, které jsou na tuto sběrnici
kladeny a které tato práce popíše a vyjde z nich při návrhu nové verze sběrnice.
Ačkoliv je sběrnice navržena pro řízení modelových kolejišť, její návrh je
obecný, takže ji lze využít v mnoha dalších aplikacích.
