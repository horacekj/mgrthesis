V~této kapitole představíme návrh vlastního \gls{plc} systému, který
splňuje požadavky definované v~\ref{subsec:gen_requirements}. Návrh systému
vychází z~přirozeného požadavku na minimální změny v~aktuálním harwaru
a~softwaru kolejiště. Nový systém je navržen jako iterace současného systému
\gls{mtb}, jmenuje se \textit{MTB v4}.

\gls{mtb} v4 řeší všechny problémy aktuálního systému \gls{mtb} popsané
v~kapitole \ref{sec:mtb_fail}. Při návrhu byly vzány v~potaz aktuální
technologie v~oblasti \gls{plc} obvodů, přičemž byla zvážena vhodnost jejich
implementace do systému \gls{mtb} v4. Tyto úvahy popíšeme. Detailně rozebereme
současnou implementaci protokolu \gls{mtbbus} a~u~každé položky budeme
argumentovat, jestli ji zachovat.

\section{Podrobná specifikace požadavků}

Pro návrh systému \gls{mtb} v4 je třeba formulovat přesnější specifikaci
požadavků, než jsou požadavky definované v~\ref{subsec:gen_requirements}. Nyní
tyto přesné požadavky formulujeme, přičemž budeme klást důraz na vysvětlení
změn oproti současnému systému \gls{mtb}.

Zaměřme se nejprve na požadavky, které na systém \gls{mtb} kladou komponenty,
které jej užívají.

\subsection{Periferie v~kolejišti}

K~\gls{mtb} se na straně kolejiště připojují jednotlivé periferie popsané
v~\ref{sec:dcc}. Všechny v~současnosti na kolejišti používané vstupy jsou
digitálního charakteru (binární vstupy), výstupy jsou buď digitální binární
nebo \textit{\gls{scom}}. \gls{scom} výstupy jsou ve skutečnosti digitální
výstupy, do kterých \gls{mtb} deska moduluje \gls{scom} signál. \gls{scom}
signál je záměrně jednoduchý pomalý signál, který lze s~přehledem vytvářet
běžným výstupním pinem procesoru nebo posuvného registru \cite{scom-specs}.

Oproti současnému \gls{mtb} by nové \gls{mtb} mělo podporovat tzv. \textit{kmitavý
výstup} s~počítačově definovanou frekvencí kmitání a~pevnou střídou. Frekvence
kmitání stačí v~řádu jednotek \textit{Hz}. Kmitavý výstup se využije pro
indikace v~pultech a~k~ovládání rozpojovačů\footnote{Rozpojovač je mechanické
zařízení pevně spojené s~kolejištěm, jehož účelem je rozpojit modelové spřáhlo
bez ručního zásahu obsluhy do kolejiště.}. I~kmitavý výstup je však z~pohledu
návrhu hardwaru jednoduchý digitální výstup – pro přidání kmitavého výstupu
stačí upravit komunikační protokol \gls{mtbbus}, software v~počítači a~firmware
modulů.

Současné moduly \gls{mtbuni} umožňují výstupy typu \gls{scom} pouze na pinech
0–7, což je omezení dané malou pamětí procesoru současných \gls{mtbuni} modulů.
Toto omezení by mělo \textit{MTB v4} relaxovat a umožnit tak, aby každý výstup
mohl být v~jednom z~režimů

\begin{compactenum}
\item digitální,
\item \gls{scom},
\item kmitavý.
\end{compactenum}

Co se týče vstupně-výstupních požadavků periferií kolejiště, \textit{MTB v4} by
si vystačilo pouze s~moduly typu \gls{mtbuni}. Na kolejištích \gls{kmz}
v~současnosti není žádný prvek, který by vyžadoval například analogový vstup,
analogový výstup nebo třeba \textit{PWM} výstup. Na kolejištích se sice hojně
používají servomotory, ale vždy pro řízení dvoupolohových prvků: výhybka,
výkolejka, mechanické návěstidlo. Všechny tyto prvky k~sobě mají řídicí
elektroniku, která interaguje se systémem \gls{mtb} pouze digitálními piny.
Typicky stačí na každý takový prvek 2~digitální výstupy \gls{mtbuni} modulu
(pro nastavení požadované polohy) a případně 2~digitální vstupy (pro detekci
koncové polohy).

Zmiňme na tomto místě, že rozhodnutí neimplementovat PWM výstupy do
\gls{mtbuni} v4 desky je koncepčním rozhodnutím autora této práce. Autor práce
razí přístup \uv{ať jedna věc dělá jeden úkol a dělá ho dobře}. Místo návrhu
všemocného \gls{mtbuni} modulu, který za pár let bude umět i uvařit kávu, se
autor rozhodl vydat cestou univerzálního kompaktního snadno sériově
vyrobitelného modulu s~prostými digitálními vstupy a výstupy.

Připomeňme, že \gls{mtbuni} desky umožňují speciální typ vstupů – IR vstupy
(viz \ref{sec:ir}). Tyto vstupy vyžadují další elektroniku navíc.  V~nových
\gls{mtbuni} v4 deskách je přímá podpora pro IR čidla zrušena. Bude navržena
samostatná deska pro buzení a vyhodnocování stavu IR čidel, která se bude
připojovat přímo k~digitálním vstupům \gls{mtbuni} desky. Ušetří se tak
elektronika na \gls{mtbuni} modulech, na kterých se IR vstupy nevyužívají,
zjednoduší se návrh desky plošných spojů \gls{mtbuni} v4 a umožní se využít IR
čidla i~s~jinými moduly, než \gls{mtb} (např. pro přímou indikaci v~pultech).
Jedna deska bude dělat jednu věc a~bude ji dělat dobře.

V~novém systému \gls{mtb} tedy budou stačit pouze moduly typu \gls{mtbuni},
které budou navíc ořezané o~přímou podporu IR čidel. I~přes to by bylo
neperspektivní si podporu jiných typů modulů zavřít volbou nevhodného
komunikačního protokolu modulů. Nový protokol sběrnice \gls{mtbbus} by měl být
navržený univerzálně pro nejrůznější možné typy desek, které v~budoucnu mohou
přijít.  Autor této práce má již v~záměru takové desky desky vytvořit, viz
\ref{sec:future}.

\subsection{Ovládání}

Druhým \uv{uživatelem} systému \gls{mtb} je řídicí software kolejiště. Na
kolejištích \gls{kmz} se v současné době využívá řídicí software
hJOP\footnote{\url{https://hjop.kmz-brno.cz/}}. Tento software běží na
operačním systému Windows a s~hardwarem kolejiště komunikuje pomocí dynamicky
linkované knihovny, která musí dodržet specifikované API \cite{hjop:rcs:web}.
Na straně řídícího softwaru jsou tedy pro návrh \gls{mtb} v4 takřka neomezené
možnosti.

Princip řízení kolejiště počítačem však definuje zásadní požadavek na
organizaci sběrnice \gls{mtbbus} v4: jednotlivé prvky jsou řízeny počítačem
a~výhradně počítačem.

Novým požadavkem je, aby v~počítači mohlo existovat více programů, které
sběrnici řídí. V~praxi totiž nastává situace, kdy jsou na jednu sběrnici
připojené systémy, které mají být z~principu řízeny různými programy: například
řízení provozu vlaků a řízení pouličního a domovního osvětlení.  Je praktické mít
oba systémy zapojené do jedné sběrnice (aby se minimalizoval počet \gls{mtb}
modulů). Dává však také smysl, aby software pro řízení provozu železnice neřídil
osvětlení, protože to není jeho účel.

Dalším příkladem z~\gls{kmz} je jedna sběrnice, na které jsou připojeny
periferie řízení tramvajové dopravy zároveň s~periferiemi řízení silniční
dopravy. Přestože oba systémy sdílí společnou sběrnici, na úrovni počítače by
tyto systémy měly řídit dva různé programy.

Jednou z~hlavních přidaných hodnot \gls{mtb} v4 je, že umožní popsané
\textit{multi-master} řízení.

\subsection{Další požadavky}

Uveďme nyní některé méně důležité požadavky na \gls{mtb} v4, jejichž naplnění
nám však výrazně zpříjemní její používání. Již teď předejmněme, že všechny tyto
požadavky se v~implementaci podařilo naplnit.

\begin{enumerate}
\item \textbf{Umožnit nalezení \gls{mtb} modulů i~po startu sběrnice.}

	V~kapitole \ref{sec:mtbbus} byl popsán striktní proces používání
	sběrnice – od vyhledání modulů až po ukončení práce se sběrnicí. Současná
	implementace tohoto procesu neumožňuje, aby počítač detekoval moduly, které
	na sběrnici přibudou za jejího chodu. Taková situace může nastat například
	když je prvně zapnut řídicí počítač a~až pak napájení sběrnice. Je
	nepříjemné muset v~takovém případě manuálně spouštět proces skenování
	\gls{mtb} modulů znovu.

	Obecně se celý postup práce se sběrnicí popsaný v~kapitole \ref{sec:mtbbus}
	zdá být zbytečně striktním. \gls{mtb} v4 by mělo tento proces zjednodušit.

\item \textbf{Z~počítače bude možné na každém modulu zapnout identifikační LED.}

	Pro snadnou identifikaci konkrétního modulu v~kolejišti, například při
	diagnostice závady, by operátor měl mít možnost z~počítače na libovolném
	modulu zapnout indikační LED. Při diagnostice pod kolejištěm tak snadno
	identifikuje konkrétní modul.

\item \textbf{Firmware \gls{mtb} modulů by mělo být možné aktualizovat přímo
	po sběrnici \gls{mtbbus}.}

	V~kolejištích \gls{kmz} je aktuálně nasazeno 99~\gls{mtb} modulů.
	Aktualizace jejich firmwaru ručním programováním všech modulů by byla
	časově náročná. \gls{mtb} v4 by mělo umožňovat aktualizaci firmwaru modulů
	přímo po \gls{mtbbus}, nejlépe za současného plného chodu zbytku modulů.

\end{enumerate}

\subsection{Shrnutí požadavků} \label{sub:mtbbus-req-summary}

Shrňme stručně všechny požadavky na systém \gls{mtb} v4.

\begin{compactenum}
\item K~\gls{mtb} v4 se musí na straně počítače dát přistoupit skrze
	\textit{hJOP RCS API}\footnote{\url{https://hjop.kmz-brno.cz/rcs}}.
\item V~počítači může běžet více softwarů řídících výstupy a snímajících vstupy
	modulů sběrnice \gls{mtbbus} v4.
\item Systém \gls{mtb} v4 musí být navržený univerzálně pro nejrůznější typy
	modulů. Jedinou společnou vlastností modulů je, že snímají vstupy a nastavují
	výstupy.
\item Je třeba vyřešit zpětnou kompatibilitu se stávajícími \gls{mtb} deskami
	v~kolejišti.
\item Celé řešení by mělo být psáno opensource a openhardware.
\item Desky sběrnice \gls{mtbbus} v4 by mělo být možné průběžně vyhledávat
	a detekovat nefunkční.
\item Implementace \gls{mtb} v4 by měla využívat osvědčené komponenty, u~nichž
	je dlouhodobý výhled dostupnosti.
\item \gls{mtb} moduly potvrzují příkazy, lze monitorovat jejich správnou funkčnost.
\item Z~počítače bude možné na každém \gls{mtb} modulu zapnout identifikační LED.
\item Moduly \gls{mtbuni} budou nově podporovat kmitavé výstupy.
\item Moduly \gls{mtbuni} budou nově podporovat \gls{scom} výstupy na všech
	výstupních pinech.
\end{compactenum}


\section{Návrh systému MTB v4}

Nyní vyjdeme z~detailních požadavku definovaných v~předchozí kapitole
a~provedeme návrh nového systému pro sběr dat a~ovládání periferií kolejiště.
Detailně zdůvodníme každý aspekt nově navrženého systému.

\subsection{Vysokoúrovňový návrh}

Základní koncept fungování systému \gls{mtb} \uv{single master, multiple
slaves} bude zachován. Bude vytvořena nová verze modulu \textit{\gls{mtbusb}},
který bude jako jediný řídicím prvkem nové sběrnice \gls{mtbbus} v4. Model
\textit{single master, multiple slaves} je osvědčeným modelem, který se snadno
implementuje, lze snadno spočítat nejlepší a nejhorší časy odezvy sběrnice a
obecně se dobře mapuje na problém, který systém \gls{mtb} řeší.

Vytvořením nové verze modulu \textit{MTB-USB} autor práce sleduje především
naplnění cíle existence svobodné implementace systému \gls{mtb}. Nová deska
umožní využití moderních procesorů a integrovaných obvodů, které usnadní
implementaci. \gls{mtbusb} deska může být vytvořena jako úplně nová, protože je
pouze jedna pro celé kolejiště a~tedy není problém ji vyměnit. Malé počty desek
implikují malé náklady na výměnu.

Bude navrhnut nový \gls{mtbuni} modul, který bude splňovat všechny požadavky
definované výše a~který bude založen na moderních, ale dlouhodobě dostupných
součástkách.

Nebude však nutné vyměnit všechny stávající \gls{mtbuni} moduly v~kolejišti.
Stávajícím modulům bude vyměněn procesor, starým deskám tak bude umožněno
fungovat na nové sběrnici. Tím se minimalizují náklady na povýšení sběrnice.

Autor navrhne a popíše IR desku, která bude zajišťovat podporu IR čidel,
protože nová \gls{mtbuni} deska již nebude obsahovat přímou podporu IR čidel.

Zachování většiny stávajícího hardwaru současných \gls{mtbuni} a snaha
o~zachování kabeláže sběrnice s~sebou nese ponechání volby sběrnice \gls{mtbbus}
na standardu RS485. RS485 nepřináší žádné zjevné nevýhody, takže není důvod
ke změně hardwarové vrstvy sběrnice.

Nové procesory v~\gls{mtb} modulech umožní trvalé uložení konfigurace v~modulech.
Přesto bude autoritativním zdrojem konfigurace modulů i~nadále počítač: po
nalezení modulu na sběrnici počítač tento modul zkonfiguruje dle konfiguračního
souboru programu. Tento autor zvolil proto, aby bylo možné  konfiguraci všech
desek kolejiště centralizovaně ukládat a verzovat.\footnote{Uložení konfigurace
v~modulech i~tak má smysl – například aby modul mohl aplikovat uložený bezpečný
stav výstupů přímo při zapnutí napájení modulu a~nemusel čekat na zapnutí
řídicího SW v~počítači.}

\gls{mtb} moduly budou skrze \gls{mtbusb} komunikovat s~počítačem po portu
\gls{usb}, kterým bude tunelován \textit{USB \gls{cdc}} protokol. V~počítači tak bude
\gls{mtbusb} modul dostupný jako sériový port. Počítač nemůže komunikovat po
sběrnici RS485 přímo, protože na to nemá hardwarové periferie, proto bude
komunikovat s~modulem \gls{mtbusb}, který bude řešit časově kritické operace
sběrnice \gls{mtbbus}\footnote{Například kontrola odpovědi modulu v~časovém
limitu cca $250~us$.}.

Otázkou zůstává, kde naplnit požadavek na \textit{multi-master} řízení celého
systému. Existují 2~možné přístupy:

\begin{compactenum}
\item \textit{MTB-USB} deska se bude k~počítači připojovat rozhraním, které
	nativně umožňuje komunikaci pouze s~jednou aplikací (např. USB), řízení více
	programy se bude řešit na úrovni softwaru v~počítači.
\item \textit{MTB-USB} deska se připojí k~počítači rozhraním, které přirozeně
	podporuje více připojených zařízení, například \textit{ethernet}.
\end{compactenum}

Oba přístupy se v~komerčních systémech pro digitální řízení modelové železnice
používají\footnote{Viz např. rozhraní \textit{Lenz LI-USB-Ethernet}, centrála
\textit{Digikeijs DR5000} aj.}.
Ať bude podpora více řídicích systému implementována kdekoliv, oproti jednomu
řídicímu systému vyžaduje tato podpora netriviální softwarovou podporu navíc.
Autor práce se rozhodl přesunout co nejvíce složitější logiky do počítačových
aplikací, protože ty se mnohem snáze vyvíjejí, ladí a aktualizují než
firmware v~embedded procesorech. V~počítači je navíc možné využívat
vysokoúrovňovější nástroje. Proto se autor vydal cestou, kdy se
\gls{mtbusb} připojuje k~počítači pomocí virtuálního sériového portu, který
umožňuje připojení jen jedné počítačové aplikace.

V~počítači vznikne nová aplikace \textit{MTB Daemon}, která se na jedné
straně připojí k~\gls{mtbusb} a~na straně druhé vystaví JSON TCP
server s~jednoduchým API, které umožní povelovat \gls{mtb} moduly z~více
počítačových programů. Získá se tím mj. hezké programové API, ke kterému
se bude dát snadno připojit z~knihoven nejrůznějších programovacích jazyků.
Tím se systém \gls{mtb} zpřístupní širšímu spektru programátorů.

Třída \gls{cdc} mezi počítačem a \gls{mtbusb} byla zvolena proto, že se jedná
de fakto o~standard pro připojení specifických periferií k~počítači, například
v~oblasti robotiky. S~výhodou využijeme toho, že protokol \gls{cdc} má nativní
podporu ovladačů v~majoritních operačních systémech a není tak nutné instalovat
speciální ovladače. Tento aspekt umožní snadné nasazení dalším uživatelům.

\subsection{\gls{mtbbus}}

Nyní podrobně rozebereme návrh nové sběrnice \gls{mtbbus} v4. Autorem vytvořená
plnohodnotná dokumentace nové sběrnice je k~dispozici na
\url{https://github.com/kmzbrnoI/mtbbus-protocol}.

Podrobné požadavky definované v~\ref{sub:mtbbus-req-summary} vynucují razantní
úpravy komunikačního protokolu sběrnice\footnote{Například umožnit odesílat
delší zprávy (kvůli aktualizaci FW \gls{mtb} modulů), navrhnout protokol
univerzálně pro libovolné typy modulů apod.}. Autor této práce se rozhodl
sběrnici navrhnout od základů znovu. Sleduje tím především zajištění
přehlednosti komunikačního protokolu a~také vyřešení potíží s~licencováním
původního protokolu. Nový protokol je dílem autora této práce a~tudíž si autor
může stanovit licenční podmínky sám.

\subsubsection{Hardware}

Jak již bylo zmíněno výše, po hardwarové stránce je zachován standard
\textit{RS485}. Počet komunikačních bitů je ponechán na 9\footnote{
V~dalším textu budeme o~jednotlivých devitibitových blocích dat mluvit jako
o~bytech, přestože se fakticky jedná o~devítibitová slova. Pro účely datového
využití zprávy je však relevantních jen 8 bitů – byte.}.
Devátý bit ve zprávách z~\gls{mtbusb} to \gls{mtb} modulu indikuje, že
se jedná o~první byte zprávy. Ostatní byty zprávy mají tento bit nastavený na
0. Devítibitová komunikace je použita s~cílem na sběrnici zavést klíčový pojem
\textit{zpráva}. Díky devátému bitu \gls{mtb} modul dokáže poznat, kdy zpráva
začíná. Počet stop bitů je ponechán standardně na jednom bitu, kontrola
konzistence dat paritou není použita.

Současná sběrnice \gls{mtbbus} podporuje různé rychlosti sběrnice –
$38400~Bd$, $57600~Bd$ a $115200~Bd$. Komunikace vždy začíná na
$38400~Bd$, jedna z~prvních zpráv sběrnice je příkaz na změnu rychlosti
sběrnice. Při výpadků a~opětovném oživení modulu musí modul postupně vyzkoušet
všechny 3~rychlosti a~sledovat, na které rychlosti přijímá korektní data
\cite{mtbbus-specs}.

Možnost volby rychlosti sběrnice by autor chtěl ponechat – některá kolejiště se
spletitým zapojením sběrnice, dlouhou sběrnicí nebo sběrnicí se špatnými elektrickými
vlastnostmi mohou vyžadovat nižší rychlosti. Bude stačit ponechat volbu mezi
třemi rychlostmi.
Rychlost nižší než $38400~Bd$ by neměla být třeba – tuto rychlost by měla
zvládat každá sběrnice. Při rychlosti $115200~Bd$ dojde ke $20$ skenům
jednoho modulu za sekundu při počtu $50$ modulů na sběrnici. Tato hodnota je
pro účely zabezpečovacího softwaru kolejiště dostatečná, takže není
třeba umožnit vyšší rychlosti.


\subsubsection{Princip komunikace}

Sběrnice \gls{mtbbus} v4 funguje v~režimu \textit{single master, multiple
slaves}. Chod sběrnice řídí \textit{master} modul – \gls{mtbusb}.
Master modul periodicky dotazuje všechny ostatní moduly sběrnice (pomocí zprávy
\textit{Module Inquiry}). \gls{mtb} modul na každý dotaz odpoví právě jednou
zprávou\footnote{Proto není nutné ve směru \textit{slave $\rightarrow$ master}
nastavovat devátý bit.}.  Pokud \gls{mtb} modul nemá žádná data k~odeslání,
odpoví zprávou \textit{Acknowledgement}. Tím \gls{mtbusb} modulu potvrdí, že
zprávu přijal a~že komunikuje. \gls{mtbusb} tak může monitorovat aktivní moduly
sběrnice a~dokonce detekovat moduly nové. Mechanismus pravidelného dotazování
modulů nazýváme \textit{polling}.

Příklad komunikace po sběrnici:

\begin{verbatim}
> Module 1 Inquiry
< Acknowledgement  # Modul 1 žije a nemá žádná data k~odeslání
> Module 2 Inquiry
# Timeout – modul 2 nežije
> Module 3 Inquiry
< State of inputs changed, new state: 0b00001010 0b11110000
> Module 6 Inquiry
< Acknowledgement
> Change outputs of module 1 to 0b00101010 0b01000101
> Module 10 Inquiry
< Acknowledgement
...
\end{verbatim}

Všimněte si, že probíhá polling pouze některých adres – pro zmenšení latencí je
vhodné často dotazovat aktivní moduly a~neaktivní moduly dotazovat jen čas od
času (aby se dalo poznat, že moduly ožily). Dále si všimněte, že příkazy pro
moduly (\textit{Change outputs}) se posílají nezávisle na tom, který modul je
právě dotazován.

\subsubsection{Komunikační protokol} \label{subsub:mtbbus-proto-strucure}

Každá \textit{zpráva} se skládá z~následujících bytů:

\begin{enumerate}
\item \textbf{Adresa modulu.}

Tento byte má jako jediný nastavený svůj 9. bit na hodnotu 1.
Adresa modulu je 8bitové číslo. Adresa 0 znamená, že zpráva se posílá
jako \textit{broadcast}, tedy ji mají číst všechny moduly na sběrnici. Typickou
zprávou typu \textit{broadcast} je požadavek na reset výstupů všech modulů.

Adresní prostor 255 modulů byl vyhodnocen jako dostatečně velký.

\item \textbf{Délka zprávy.}

Tento byte obsahuje počet \textit{datových bytů zprávy} plus jedna (pro
\textit{kód zprávy}).

Pro univerzálnost protokolu bylo rozhodnuto jej navrhnout tak, aby délka zpráv
nebyla součástí definice jednotlivých zpráv, ale aby byla ve zprávě obsažena
explicitně. Toto rozhodnutí umožňuje například \gls{mtbusb} modulu
přeposílat zprávy mezi \gls{mtbbus} a~počítačem, aniž by musel dekódovat obsah
zpráv, případně mít uloženou tabulku \textit{typ zprávy, délka}. Některé zprávy
navíc mohou mít variabilní délku ze své podstaty.

Maximální hodnota tohoto bytu je 121, což umožňuje zprávy skládat do bufferů
o~délce 128 bytů. Maximální délka zprávy byla zvolena s~ohledem na nevelké
kapacity \textit{SRAM} mikrokontrolérů a~časovou efektivitu případné
retransmise.

Současná verze protokolu \gls{mtbbus} umožňuje zprávy o~délce nejvýše 7 bytů
\cite{mtbbus-specs}.  Tato hodnota byla zásadně navýšena zejména proto, aby
bylo možné novým protokolem posílat firmware pro aktualizaci \gls{mtb} modulů.

\item \textbf{Kód zprávy}

Libovolné číslo v~rozsahu 0–255, definuje význam zprávy. Význam jednotlivých
kódů definuje specifikace protokolu.

\item \textbf{Data zprávy}

Až 120 bytů.

\item \textbf{Kontrolní součet} (2 byty)

Kontrolní součet se ve zprávě uvádí, aby bylo možné ověřit integritu přijímané
zprávy. Integrita zprávy může být narušena například elektromagnetickým rušením
nebo špatnými elektrickými vlastnostmi přenosového média.

Původní protokol \gls{mtbbus} zabezpečoval integritu zprávy pomocí jednoduchého
mechanismu \texttt{xor}, kdy na jednotlivé byty zprávy byl aplikován bitový xor
a výsledná hodnota byla připojena na konec zprávy. Tento způsob je vhodný pro
malé zprávy.

Nová verze protokolu umožňuje delší zprávy a~tak byl navržen robustnější
způsob zabezpečení zprávy. Sběrnice \gls{mtbbus} v4 používá pro zabezpečení
zprávy standardní mechanismus \textit{CRC-16}. Tento mechanismus je
založen na dělení polynomů se zbytkem. Konkrétně se používá verze
\texttt{CRC-16-IBM}. Inspirací autorovi byla průmyslová sběrnice
\textit{ModBus}, která tento způsob zabezpečení zpráv používá také
\cite{modbus-specs}.

\end{enumerate}

Příklad zprávy: \texttt{0x105 0x001 0x001 0x0D0 0x051}.

\begin{compactenum}
\item \texttt{0x105}: zpráva pro modul s~adresou 5.
\item \texttt{0x001}: následuje 1 byte zprávy.
\item \texttt{0x001}: kód zprávy \texttt{0x01}.
\item \texttt{0x0D0}, \texttt{0x051}: kontrolní součet \textit{CRC-16}.
\end{compactenum}

Zpráva neobsahuje žádná \textit{data zprávy}.

\subsubsection{Zprávy} \label{subsub:mtbbus-messages}

Kompletní specifikace významů a kódů zpráv je dostupná na
\url{https://github.com/kmzbrnoI/mtbbus-protocol}, zde uvedeme stručný přehled.

Zprávy master $\rightarrow$ slave:

\begin{compactitem}
\item \textit{Module Inquiry}
\item \textit{Module Information Request}
\item \textit{Set Configuration}
\item \textit{Get Configuration}
\item \textit{Beacon}
\item \textit{Get Input}
\item \textit{Set Output}
\item \textit{Reset Outputs}
\item \textit{Change Address}
\item \textit{Change Speed}
\item \textit{Firmware Upgrade Request}
\item \textit{Firmware Write Flash}
\item \textit{Firmware Write Flash Status Request}
\item \textit{Module-specific Command}
\item \textit{Reboot}

\end{compactitem}

Zprávy slave $\rightarrow$ master:

\begin{compactitem}
\item \textit{Acknowledgement}
\item \textit{Error}
\item \textit{Module Information}
\item \textit{Module Configuration}
\item \textit{Input Changed}
\item \textit{Input State}
\item \textit{Output Set}
\item \textit{Firmware Write Flash Status}
\item \textit{Module-specific command}
\end{compactitem}

Rozeberme nyní zprávy detailněji.

\gls{mtbusb} průběžně skenuje aktivní i~neaktivní moduly (aktivní častěji)
a~posílá modulům \textit{Module Inquiry}. Moduly odpovídají buď
\textit{Ack\-now\-led\-ge\-ment} nebo \textit{Input Changed}.

Pomocí zpráv \textit{Module Information Request} a~\textit{Module Information}
lze zjistit základní informace o~\gls{mtb} modulu: jeho typ, verzi firmwaru,
podporovanou verzi protokolu a~jestli je modul v~bootloaderu.

Zpráva \textit{Set Output} slouží k~nastavení stavu výstupů modulu, zpráva
\textit{Reset Outputs} slouží k~navrácení výstupů do základního stavu.
Tato zpráva je typicky odesílána \gls{mtbusb} modulem při ukončení komunikace
s~počítačem jako broadcast.

Počítač může zapisovat nebo vyčítat konfiguraci modulů pomocí zpráv
\textit{Get Configuration}, \textit{Set Configuration} a odpovědi
\textit{Configuration}.

Počítač může zapínat a vypínat indikační LED na \gls{mtb} modulech pomocí
zprávy \textit{Beacon}.

Modulům, které si ukládají adresu softwarově, může operátor z~počítače změnit
adresu pomocí zprávy \textit{Change Address}.

Plánovaná změna rychlosti sběrnice se modulům ohlašuje pomocí zprávy
\textit{Change Speed}, která je typicky odesílána jako broadcast. Každý modul
si ukládá rychlost sběrnice do nevolatilní paměti, uloženou rychlost použije při
zapnutí modulu. Očekává se, že budoucí typy procesorů v~modulech \gls{mtb} budou
schopny detekovat rychlost sběrnice automaticky.

V~současnosti není na sběrnici \gls{mtbbus} implementován žádný mechanismus pro
automatickou detekci rychlosti.  V~situaci, kdy je například do sběrnice
doplněn nový modul, který je ale z~výroby nastaven na jinou rychlost, než
aktuálně sběrnice používá, se očekává, že operátor přenastaví rychlost sběrnice
na rychlost nového modulu, odešle novému modulu příkaz na změnu rychlosti
a~vrátí rychlost sběrnice zpět. Tyto kroky lze provést za plného chodu sběrnice.

Popisu procesu aktualizace firmwaru modulů se budeme věnovat v~samostatné
kapitole.

Poslední zajímavou zprávou je \textit{Module-specific command}. Tato zpráva
umožňuje odeslat \gls{mtb} modulu libovolná data (a~modul může odpovědět
libovolnými daty). \textit{Module-specific command} se používá například pro
řízení specializovaných operací modulů, viz \ref{sec:mtb-2-avr}.


\subsubsection{Dvouvrstvost protokolu}

Specifikace \gls{mtbbus} v4 je tzv. \textit{dvouvrstvá}. Protokol definuje
typy zpráv, ale data některých zpráv specifikují až konkrétní typy \gls{mtb}
modulů. Typ modulu definuje:

\begin{enumerate}
\item \textbf{Formát dat vstupů, formát dat výstupů}

Protokol je navržený tak, aby moduly mohly mít libovolné typy a počty vstupů
i~výstupů. Obsah datových bytů zpráv \textit{Set Output}, \textit{Input
Changed} a~\textit{Input State} je proto specifikovaný až ve specifikaci modulů.

Například modul \gls{mtbuni} má 16~digitálních vstupů a~16~digitálních výstupů.
Protože počet vstupů i výstupů je malý, posílá se vždy celý stav
vstupů i~výstupů.

Protokol počítá s~tím, že jiné moduly mohou mít vstupy a~výstupy například
s~větším rozsahem hodnot (analogové vstupy). V~takovém případě protokol
umožňuje například neposílat celý stav vstupů a~výstupů, ale jen relevantní
část. To je také důvodem, proč existují 2 různé zprávy
\textit{Input Changed} (ve které se posílá jen stav změněných vstupů)
a~\textit{Input State} (ve které se posílá stav všech vstupů).
\footnote{Všimněte si, že tento mechanismus umožňuje provádět polling
vstupů místo událostního hlášení změn stavů vstupů. To je zcela záměrný prvek
\gls{mtbbus} v4. Využije se například u~budoucích modulů s~analogovými vstupy.}

\item \textbf{Formát konfigurace}

Konfigurace různých typů modulů se mohou zásadně lišit v~počtu
konfigurovatelných hodnot.

Například modul \gls{mtbuni} ma 26 konfiguračních bytů. Proto se celá
konfigurace posílá najednou. Protokol však umožňuje libovolný způsob
zasílání konfigurace: počítač si například u~modulů s~větším množstvím konfiguračních
bytů může vyžádat pouze podmnožinu konfiguračních hodnot. Nebo naopak poslat
podmnožinu konfiguračních adres a~hodnoty na těchto adresách k~nastavení.

\item \textbf{Adresování a data paměti pro aktualizaci firmwaru}

Různé typy modulů typicky obsahují různé procesory, které jinak adresují paměť.
Liší se například šířka slova nebo velikost paměti. Význam datových bytů
zpráv pracujících s~aktualizací firmwaru procesoru je proto definován pro
konkrétní typy modulů.

\end{enumerate}

\subsubsection{Typické fungování sběrnice}

Následující kroky popisují typické fungování systému \gls{mtb} v4 (srovnejte
s~postupem definovaným na konci \ref{sec:mtbbus}).

\begin{compactenum}
\item Je zapnut řídicí počítač kolejiště.
\item Je zapnuto napájení \gls{mtbusb} modulu a~všech \gls{mtb} modulů.
\item \gls{mtb} moduly z~paměti načtou konfiguraci, aplikují základní stav
výstupů, začnou poslouchat na sběrnici.
\item \gls{mtbusb} detekuje připojené moduly a zapamatuje si je.
\item Počítačová aplikace se připojí k~\gls{mtbusb}.
\item Počítačová aplikace vyčte z~\gls{mtbusb} seznam aktivních modulů.
\item Aplikace vyčte typ připojených modulů, zkonfiguruje je.
\item Řídicí SW kolejiště používá sběrnici – čte vstupy, nastavuje výstupy.
\item Při odpojení řídicího SW kolejiště je \gls{mtb} modulům poslán příkaz
	k~resetu stavu výstupů, čímž je zajištěno, že při odpojení nebo pádu
	řídicího softwaru se budou nacházet výstupy v~definovaném stavu.
\end{compactenum}

\gls{mtb} v4 však nevyžaduje striktní pořadí těchto kroků. Napájení
komponent může být zapnuto v~libovolném pořadí, systém dokáže detekovat výpadky
jednotlivých prvků a obnovovat je. Pro ilustraci uveďme několik příkladů situací,
které v~systému \gls{mtb} mohou nastat.

\begin{itemize}
\item Jiné pořadí zapnutí prvků.

Pokud je \gls{mtbusb} zapnut až po zapnutí počítače, počítač tento stav
korektně detekuje (chybí zařízení) a~po připojení \gls{mtbusb} jej korektně
inicializuje.

Sken všech adres sběrnice \gls{mtbbus} (255 adres) probíhá neustále, takže
pokud jsou moduly sběrnice zapnuty až po zapnutí \gls{mtbusb}, jsou korektně
detekovány.

\item Výpadek \gls{mtb} modulu.

Pokud \gls{mtb} modul neodpovídá na zprávy, \gls{mtbusb} situaci detekuje,
nahlásí ji do počítače a~počítač zareaguje.

Pokud \gls{mtbusb} detekuje nový \gls{mtb} modul, opět o~tom zpraví
počítač, ten vyčte typ desky a~zkonfiguruje ji.

\item Výměna modulu za chodu sběrnice.

Vyňatý modul je při odpojení napájení nahlášen jako neaktivní, po výměně je
aktivován nový modul.

\item Změna adresy modulu za chodu sběrnice.

Původní adresa se stane neaktivní (neodpovídá na požadavky), nová adresa
je nalezena a zkonfigurována.

\end{itemize}


\subsubsection{Aktualizace firmwaru \gls{mtb} modulů} \label{subsub:fw_uprage}

Sběrnice \gls{mtbbus} je navržena tak, aby skrze ní mohl být aktualizován
firmware \gls{mtb} modulů. Při návrhu mechanismu aktualizace firmwaru bylo
zvažováno, jestli aktualizaci nevést úplně jiným protokolem.
\gls{mtbbus} je však dostatečně obecný a navíc je třeba, aby
při aktualizaci jednoho modulu ostatní moduly sběrnice aktualizační příkazy
ignorovaly a nerušily tak aktualizaci. Tyto argumenty vedly k~implementaci
mechanismu aktualizace firmwaru přímo do sběrnice \gls{mtbbus}. Výhodou tohoto
řešení je, že zbylé moduly sběrnice mohou při aktualizaci firmwaru jednoho
modulu nadále komunikovat, aktualizace může probíhat za plného provozu.

Protokol počítá s~tím, že přepis flash paměti procesorů \gls{mtb} modulů může
probíhat pouze z~bootloaderu. Aktualizační protokol proto podporuje reboot
procesoru do bootloaderu. Procedůra aktualizace firmwaru probíhá následovně
\footnote{Procedůra je podrobně popsaná v~dokumentaci protokolu
\gls{mtbbus} na
\url{https://github.com/kmzbrnoI/mtbbus-protocol/blob/master/workflows.md}.}.

\begin{compactenum}
\item \gls{mtbusb} vyšle \textit{Firmware Upgrade Request}. Tím informuje \gls{mtb}
	modul, že má rebootovat do bootloaderu.
\item \gls{mtb} odpoví \textit{Acknowldgement} a~rebootuje do bootloaderu.
\item \gls{mtbusb} pošle \gls{mtb} modulu žádost o~jeho obecné informace.
\item \gls{mtb} odpoví obecnými informacemi, ve kterých je mj. zaznačeno,
	že je v~bootloaderu.
\item \gls{mtbusb} vyšle příkaz \textit{Firmware Write Flash}, ve kterém pošle
	blok flash paměti k~zapsání a její adresu. \gls{mtb} zapíše data do své flash
	paměti.
\item \gls{mtbusb} se zeptá \gls{mtb} modulu příkazem \textit{Firmware Write
	Flash Status Request}, jestli již byl zápis do flash paměti dokončen.
	Jakmile je zápis dokončen, přesouvá se na zápis dalšího bloku paměti (krok~5).
\item Po zapsání celého firmwaru \gls{mtbusb} pošle příkaz \textit{Reboot}.
	\gls{mtb} modul provede kontrolu konzistence paměti a~naběhne do hlavního
	programu.
\end{compactenum}

Je doporučeno (a nový firmware \gls{mtbuni} modulů toto doporučení dodržuje),
aby při každém startu procesoru \gls{mtb} modulu proběhla kontrola konzistence
paměti programu.

Kontrola konzistence v~\gls{mtbuni} v4 modulech je implementována pomocí
uloženého kontrolního součtu (\textit{CRC-16}) na konkrétních adresách paměti.
Protokol \gls{mtbbus} obsahuje atribut, kterým může \gls{mtb} modul oznámit, že
sice poslouchá, ale je v~bootloaderu a~odmítá nabootovat, protože nesedí
kontrolní součet hlavního programu.

Aktualizace bootloaderu přes sběrnici \gls{mtbbus} není možná, protože vlivem
výpadku napájení při aktualizaci by mohlo dojít k~situaci, kdy se modul stane
nepoužitelným. Bootloader by měl být malý a~odzkoušený kus firmwaru, který se
nahraje jednou a~nikdy ho nebude třeba přehrávat. Přehrání bootloaderu vyžaduje
ruční obcházení modulů s~programátorem.
