Našim základním cílem v této kapitole bude navrhnout \gls{plc} systém, který
splňuje požadavky definované na konci kapitoly \ref{chap:nasazeni}. Při návrhu
tohoto systému vyjdeme z přirozeného požadavku na minimální změny v aktuálním
harwaru a softwaru kolejiště. Navrhneme nový systém jako iteraci současného
systému \gls{mtb}. Tento systém pojmenujeme \textit{MTB v4}.

Nový systém vyřeší všechny problémy aktuální verze systému \gls{mtb} popsané
v kapitole \ref{chap:nasazeni}. Při návrhu vezmeme v potaz aktuální technologie
v oblasti \gls{plc} obvodů a zvážíme implementace těcht technolgií. Detailně
rozebereme současnou implementace protokolu \gls{mtbbus} a u každé položky
zvolíme, jestli a proč ji chceme zachovat.

\section{Podrobná specifikace požadavků}

Abychom byli schopni navrhnout MTB v4, musíme znát přesné požadavky na takový
systém. Nyní požadavky formulujeme, přičemž budeme klást důraz na vysvětlení
změn oproti současnému systému \gls{mtb}.

Zaměřme se nejprve po požadavky, které na systém \gls{mtb} kladou komponenty,
které tento systém používají.

\subsection{Periferie v kolejišti}

K \gls{mtb} se na straně kolejiště připojují jednotlivé vstupy a výstupy
popsané v kapitole \ref{chap:existujici-reseni}. Všechny v současnosti na
kolejišti používané vstupy jsou digitálního formátu (binární vstupy), výstupy
jsou buď digitální binární nebo S-COM \ref{scom}. S-COM výstupy jsou ve
skutečnosti digitální výstupy, do kterých MTB deska moduluje S-COM signál.
S-COM signál je jednoduchý pomalý signál, který lze s přehledem vytvářet běžným
výstupním pinem procesoru nebo posuvného registru.

Oproti současnému \gls{mtb} bychom nově chtěli podporovat \textit{kmitavý
výstup} s počítačově definovou frekvencí kmitání a pevnou střídou. Jednalo by
se o kmitání v řádu jednotek \textit{Hz}, využití tohoto typu výstupu je pro
indikace v pultech a rozpojovače \ref{rozp}. I toto je však z pohledu hardwaru
jednoduchý digitální výstup, pro přidání tohoto typu výstupu je třeba upravit
pouze komunikační protokol a software a firmware modulů.

Současné moduly \gls{mtbuni} umožňují výstupy typu S-COM pouze na pinech
0–7, což je omezení dané malou pamětí procesoru. Toto omezení by mělo
\textit{MTB v4} relaxovat a umožnit tak, aby každý výstup mohl být v jednom
z režimů

\begin{compactenum}
\item digitální,
\item S-COM,
\item kmitavý.
\end{compactenum}

Co se týče vstupně-výstupních požadavků, \textit{MTB v4} by si vystačilo pouze
s moduly typu \gls{mtbuni} \ref{uni}. Na kolejišti v současnosti není žádný
prvek, který by vyžadoval například analogový vstup, analogový výstup nebo např.
\textit{PWM} výstup. Na kolejištích se hojně používají servomotory, ale vždy
pro řízení dvoupolohových prvků: výhybka, výkolejka, mechanické návěstidlo.
Všechny tyto prvky k sobě mají řídicí elektroniku, která interaguje se systémem
\gls{mtb} pouze digitálními piny. Typicky stačí 2 digitální výstupy (pro
nastavení požadované polohy) a případně 2 digitální vstupy (pro detekci koncové
polohy).

Zmiňme na tomto místě, že neimplementace PWM výstupů do \gls{mtbuni} v4
desky je koncepčním rozhodnutím autora této práce. Autor práce razí přístup
\uv{ať jedna věc dělá jeden úkol a dělá ho dobře}. Místo návrhu všemocného
\gls{mtbuni} modulu, který za pár let bude umět i uvařit kávu, se autor
rozhodl vydat cestou univerzálního kompaktního snadno sériově vyrobitelného
modulu s prostými digitálními vstupy a výstupy.

Připomeňme, že \gls{mtbuni} desky umožňují speciální typ vstupů – IR vstupy,
viz sekci \ref{sec:uni_ir}. Tyto vstupy vyžadují další elektroniku navíc.
V nových \gls{mtbuni} v4 deskách bude podpora pro IR čidla zrušena ve jménu
předchozího odstavce. Bude navržena speciální deska pro buzení a vyhodnocování
stavu IR čidel, která se bude připojovat přímo k digitálním vstupům
\gls{mtbuni} desky. Ušetří se tak elektronika na \gls{mtbuni} deskách,
na kterých se IR vstupy nevyužívají, zjednoduší se návrh desky plošných spojů
a umožní se využít IR čidla i s jinými deskami, než MTB, např. pro přímou
indikaci v pultech. Jedna deska bude dělat jednu věc a bude ji dělat dobře.

Na současném kolejišti si tedy vystačím pouze s \gls{mtbuni} deskami, které
budou navíc ořezané o podporu IR čidel. Ačkoliv je toto tvrzení pravdivé, bylo
by neperspektivní si podporu jiných typů desek zavřít volbou nevhodného
protokolu. Nový protokol sběrnice \gls{mtbbus} by měl být navržený univerzálně
pro nejrůznější možné typu desek, které v budoucnu mohou přijít. Autor této
práce má již v záměru takové desky desky vytvořit, více se můžete dočíst
v budoucí kapitole \ref{todo}.

\subsection{Ovládání}

Dalším uživatelem systému \gls{mtb} je řídicí software kolejiště. Tento
software je aktuálně na programován pro operační systém Windows a s hardwarem
kolejiště komunikuje pomocí dynamicky linkované knihovny, která musí dodržet
specifikované API \cite{api}. Na straně softwaru máme tedy relativně volné
ruce.

Princip řízení kolejiště dává požadavek na organizaci \gls{mtbbus} v4 sběrnice:
jednotlivé prvky jsou řízeny počítačem a výhradně počítačem.

Novým požadavkem je, aby v počítači mohlo existovat více softwarů, které
sběrnici řídí. V praxi totiž nastává kdy na jednu sběrnici jsou připojené systémy,
které mají být z principu řízeny různými programy: například zabezpečení řízení
provozu kolejiště a řízení pouličního a domovního osvětlení. Je vhodné mít oba
systémy zapojené do jedné sběrnice, aby se minimalizoval počet modulů, zejména
proto, že výstupů na řízení osvětlení je oproti řízení zabezpečení provozu málo.
Dává však také smysl, aby software pro řízení provozu neřídil osvětlení, protože
to není jeho účel.

Dalším reálným příkladem z \gls{kmz} je jedna sběrnice, na které jsou připojeny
periferie řízení tramvajové dopravy zároveň s periferiemi řízení silniční
dopravy. Přestože oba systémy sdíli společnou sběrnici, na úrovni počítače by
tyto systémy měly řídit dva různé programy. Jednom z hlavních přidaných hodnot
návrhu nového systému MTB je, že umožní \textit{multi-mater} řízení.

\subsection{Další požadavky}

Uveďme nyní některé méně důležité požadavky na novou verzi sběrnici \gls{mtbbus},
jejichž naplnění nám však výrazně zpříjemní její používání. Již teď předejímáme,
že všechny tyto požadavky se v implementaci podařilo naplnit.

\begin{enumerate}
\item \textbf{Umožnit nalezení \gls{mtb} modulů i po startu sběrnice.}

	V kapitole \ref{sec:mtbbus} byl popsán poměrně striktní proces používání
	sběrnice – od vyhledání modulů až po ukončení práce se sběrnicí. Současná
	implementace tohoto procesu neumožňuje, aby počítač detekoval moduly, které
	na sběrnici přibudou za jejího chodu. Taková situace může nastat například
	tak, že je prvně zapnut řídicí počítač a až pak napájení sběrnice. Je
	nepříjemné muset v takovém případě manuálně spouštět proces skenování
	znovu.

	Obecně se celý postup práce se sběrnicí popsaný v kapitole \ref{sec:mtbbus}
	zdá být příliš striktním. \gls{mtb} v4 by mělo tento proces zjednodušit.

\item \textbf{Z počítače bude možné na každém modulu zapnout identifikační LED.}

	Pro snadnou identifikaci konkrétního modulu v kolejišti, například při
	diagnostice závady, by operátor měl mít možnost na libovolném modulu
	zapnout indikační LED. Při diagnostice pod kolejištěm tak snadno identifikuje
	konkrétní modul.

\item \textbf{Firmware \gls{mtb} modulů by mělo být možné aktualizovat přímo
	po sběrnici \gls{mtbbus}.}

	Pod kolejištěm jsou nasazeny vyšší desítky modulů. Aktualizace jejich
	firmwaru ručním programováním všech modulů by byla časově náročná. Sběrnice
	by měla umožňovat aktualizaci firmwaru modulů přímo, ideálně za současného
	plného chodu zbytku modulů.

\end{enumerate}

\subsection{Shrnutí požadavků}

Shrňme stručně všechny požadavky na systém \gls{mtb} v4.

\begin{compactenum}
\item K \gls{mtb} v4 se musí na straně počítače dát přistoupit skrze hJOP RCS
	API \cite{}.
\item V počítači může běžet více softwarů řídících výstupy a snímajících vstupy
	modulů sběrnice \gls{mtbbus} v4.
\item Systém \gls{mtb} v4 musí být navržený univerzálně pro nejrůznější typy
	modulů. Jedinou společnou vlastností modulů je, že snímají vstupy a nastavují
	výstupy.
\item Je třeba vyřešit zpětnou kompatibilitu především se stávajícími
	\gls{mtbuni} deskami v kolejišti.
\item Celé řešení by mělo být psáno opensource a openhardware, aby se dalo volně
	využívat.
\item Desky sběrnice \gls{mtbbus} v4 by mělo být možné průběžně vyhledávat,
	a detekovat nefunkční.
\item Implementace systému by měla využívat osvědčené komponenty, u nichž
	je dlouhodobý výhled dostupnosti.
\item Desky \gls{mtb} potvrzují příkazy, lze monitorovat jejich správnou funkčnost.
\item Z počítače bude možné na každém modulu zapnout identifikační LED.
\item Moduly \gls{mtbuni} budou nově podporovat kmitavé výstupy.
\item Moduly \gls{mtbuni} budou nově podporovat S-COM výstupy na všech pinech.
\end{compactenum}


\section{Návrh systému MTB v4}

Nyní provedeme návrh nového systému pro sběr dat z kolejiště a ovládání
periferií kolejiště. Detailně zdůvodníme každý aspekt nově navrhnutého systému.

\subsection{Vysokoúrovňový návrh}

Zachováme základní koncept fungování systému \gls{mtb} \uv{single master,
multiple slaves}. Vytvoříme novou verzi modulu \textit{MTB-USB}, který bude
jako jediný řídicím prvkem nové sběrnice \gls{mtbbus} v4. Model \textit{single
master, multiple slaves} je osvědčeným modelem, který se snadno implementuje,
lze snadno spočítat nejlepší a nejhorší časy odezvy sběrnice a obecně se dobře
mapuje na problém, který systémem \gls{mtb} chceme řešit.

Vytvořením nové verze modulu \textit{MTB-USB} sledujeme především to, aby
existovala svobodná implementace této desky, nová deska nám dále umožní využívat
moderních procesorů a integrovaných obvodů, které nám usnadní implementaci.
Tuto desku si můžeme dovolit vytvořit úplně novou, protože je pouze jedna
pro celé kolejiště. Malé počty desek implikují malé náklady na aktualizaci
sběrnice, což je přesně to, co chceme.

Navrhneme novou \textit{MTB-USB} desku, která bude splňovat všechny požadavky
definované výše a která bude založena na moderních, ale dlouhodobě dostupných
součástkách.

Nebudeme však vynucovat výměnu všech stávajících \gls{mtbuni} desek v
kolejišti. Provedeme výměnu procesoru v současné \gls{mtbuni} desce za
procesor nový a umožníme tak starým deskám fungovat na nové sběrnici. Tím se
minimalizují náklady na povýšení sběrnice.

Navrhneme IR desku, která bude zajišťovat podporu IR čidel, protože nová
\gls{mtbuni} deska již nebude obsahovat přímou podporu IR čidel.

Zachování většiny stávajícího hardwaru současných \gls{mtbuni} a snaha o
zachování kabeláže sběrnice s sebou nese ponechání volby sběrnice \gls{mtbbus}
na standardu RS485. RS485 nepřináší žádné zjevné nevýhody, takže není důvod,
proč hardwarovou vrstvu komunikační sběrnice měnit.

Nové procesory v \gls{mtb} modulech umožní trvalé uložení konfigurace v modulech.
Přesto bude autoritativním zdrojem konfigurace modulů i nadále počítač: po
nalezení modulu na sběrnici počítač tento modul zkonfiguruje dle konfiguračního
souboru. Tento přístup volíme proto, abychom byli schopni konfiguraci všech
desek kolejiště centralizovaně ukládat a verzovat.\footnote{Uložení konfigurace
v modulech i tak má smysl – například aby modul mohl aplikovat uložený bezpečný
stav výstupů přímo při zapnutí napájení modulu a nemusel čekat na zapnutí
řídicího SW v počítači.}

\gls{mtb} moduly budou skrze desku \textit{MTB-USB} komunikovat s počítačem
po portu USB, kterým bude tunelován \textit{USB CDC} protokol. V počítači tak
bude celá sběrnice přístupná skrze sériový port. Počítač nemůže komunikovat
po sběrnici RS485 přímo, protože na to nemá hardwarové periferie, proto bude
komunikovat s modulem \textit{MTB-USB}, který bude řešit časově kritické operace
sběrnice \gls{mtbbus}. \footnote{TODO popsat?}

Otázkou zůstává, kde naplníme požadavek na \textit{multi-master} řízení celého
systému. Existují v zásadě 2 možné přístupy:

\begin{compactenum}
\item \textit{MTB-USB} deska se bude k počítači připojovat rozhraním, které
	nativně umožňuje komunikaci pouze s jednou aplikací (např. USB) a řízení více
	programy se bude řešit na úrovni softwaru v počítači.
\item \textit{MTB-USB} deska se připojí k počítači rozhraním, které přirozeně
	podporuje více připojených zařízení, například ethernet.
\end{compactenum}

Oba přístupy se v komerčních systémech pro digitální řízení modelové železnice
používají. Ať bude podpora více řídicích systému implementována kdekoliv, oproti
jednomu řídicímu systému vyžaduje tato podpora netriviální programovou podporu
navíc. Autor práce se rozhodl přesunout co nejvíce složitější logiky do počítačových
aplikací, protože se ty se mnohem snáze vyvíjejí, ladí a aktualizují než firmware
v embedded procesorech. V počítači můžeme navíc využívat vysokoúrovňovější
nástroje. Vydáme se tedy cestou, kdy se deska \textit{MTB-USB} bude v k počítači
připojovat již pomocí zmíněného virtuálního sériového portu. V počítači
naprogramujeme novou aplikaci \textit{MTB Daemon}, která se na jedné straně
připojí k \textit{MTB-USB} modulu a na straně druhé vystaví JSON TCP server
s jednoduchým API, které umožní povelovat \gls{mtb} moduly z více počítačových
programů. Získáme tím mj. velice hezké programové API, ke kterému se bude dát
snadno připojit z knihoven nejrůznějších programovacích jazyků, a tím
zpřístupníme systém \gls{mtb} širšímu spektru programátorů.

Protokol CDC mezi počítačem a \textit{MTB-USB} deskou zvolíme proto, že se
jedná de fakto o standard pro připojení specifických periferií k počítači,
například v oblasti robotiky. S výhodou využijeme toho, že protokol CDC má
nativní podporu ovladačů snad v majoritních operačních systémech a není tak
nutné instalovat speciální ovladače. Tento aspekt nám umožní budoucí snadné
nasazení u externích zákazníků.

\subsection{\gls{mtbbus}}

Nyní podrobně rozebereme návrh nové sběrnice \gls{mtbbus}. Oficiální autorem
vytvořená plnohodnotná dokumentace nové sběrnice je k dispozici na
\url{https://github.com/kmzbrnoI/mtbbus-protocol}.

Požadavky definované v předchozích kapitolách vynucují razantní úpravy
komunikačního protokolu sběrnice. Autor této práce se rozhodl sběrnici
navrhnou celou od píky znovu. Sleduje tím především zajištění
\textit{přehlednosti} komunikačního protokolu sběrnice a také vyřešení potíží s
licencováním protokolu. Nový protokol je autorským dílem autora této práce
a tudíž si licenční podmínky může stanovovat sám.

\subsubsection{Hardware}

Jak již bylo zmíněno v předchozí sekci, po hardwarové stránce bude zachován
standard \textit{RS485} \cite{rs485}. Počet komunikačních bitů bude ponechán na 9.
Devátý bit ve zprávách z master to slave desky indikuje, že se jedná o první
byte zprávy. Ostatní bytes zprávy mají tento bit nastavený na 0. Devítibitová
komunikace nám umožní na sběrnici zavést klíčový pojem \textit{zprávy}. Díky
devátému bitu slave modul dokáže poznat, kdy zpráva začíná. Počet stop bitů
ponecháme standardně na jednom bitu, paritní bit nepoužijeme.

Aktuální specifikace sběrnice \gls{mtbbus} podporuje různé rychlost sběrnice –
38400 Bd, 57600 Bd a 115200 Bd. Komunikace vždy začíná na 38400 Bd a jeden
z prvních příkazů je příkaz na změnu rychlost sběrnice. Při výpadků a opětovném
oživení modulu musí modul postupně vyzkoušet ony 3 rychlost a sledovat, na které
rychlosti přijímá korektní data.

Možnost různých rychlostí by autor chtěl ponechat – některé kolejiště se
spletitou sběrnicí, dlouhou sběrnicí nebo sběrnicí se špatnými elektrickými
vlastnostmi mohou vyžadovat nižší rychlosti. Rchlost nižší než 38400 Bd by
neměla být třeba – tuto. rychlost by měla zvládat každá sběrnice. Při rychlosti
115200 Bd dojde průměrně k 10 skenům jednoho modulu při počtu 50 modulů.
Tato hodnota je poměrně malá, ale pro účely zabezpečovacího softwaru kolejiště
dostatečná. V budoucnu možná dojde k dalšímu zvýšení rychlosti sběrnice.


\subsubsection{Princip komunikace}

Sběrnice bude fungovat v režimu \textit{single master, multiple slaves}. Chod
na sběrnice bude řídit \textit{master} module – \textit{MTB-USB}. Master modul
bude periodicky dotazovat všechny moduly sběrnice s výjimkami zpráv pro konkrétní
moduly odeslané z počítače. Modul na každou zprávu pro něj odpoví právě jednou
zprávou. Proto není nutné ve směru \textit{slave $\rightarrow$ master} používat
devátý bit.

Modul na periodické výzvy o odpověď (\textit{Module Inquiry}) vždy odpoví daty
k odeslání. Pokud modul nemá žádná data k odeslání, odpoví zprávou
\textit{Acknowledgement}. Tím \textit{MTB-USB} modulu potvrdí, že zprávu přijal
a že komunikuje. \textit{MTB-USB} modul tka může monitorovat aktivní moduly
sběrnice a dokonce detekovat moduly nové.

Příklad komunikace po sběrnici:

\begin{verbatim}
> Module 1 Inquiry
< Acknowledgement  # Module 1 žije a nemá žádná data k odeslání
> Module 2 Inquiry
# Timeout – modul 2 nežije
> Module 3 Inquiry
< State of inputs changed, new state: 0b00001010 0b11110000
> Module 6 Inquiry
< Acknowledgement
> Change outputs of module 1 to 0b00101010 0b01000101
> Module 10 Inquiry
< Acknowledgement
...
\end{verbatim}

Všimněte si, že v příkladu probíhá polling pouze některých adres – pro zmenšení
latencí je vhodné často dotazovat aktivní moduly a neaktivní moduly skenovat
jen čas od času (aby se dalo poznat, že moduly ožily). Dále si všimněte, že
příkazy pro moduly chodí nezávisle na tom, která adresa je právě dotazována.

\subsubsection{Komunikační protokol} \label{subsub:mtbbus-proto-strucure}

Každá \textit{zpráva} má následující strukturu:

\begin{enumerate}
\item \textbf{Adresa modulu.}

Tento byte má jako jediný nastavený svůj 9. bit na hodnotu 1.
Adresa modulu je 8bitové číslo. Adresa 0 znamená, že příkaz se posílá
jako \textit{broadcast} a mají ho číst všechny moduly na sběrnici. Typickým
příkazem typu \textit{broadcast} je požadavek na reset výstupů všech modulů.

Adresní prostor 255 modulů byl vyhodnocen jako dostatečně velký.

\item \textbf{Délka zprávy.}

Tento byte obsahuje počet datových bytů zprávy plus jedna. Pro univerzálnost bylo
rozhodnuto protokol navrhnout tak, aby délka zpráv nebyla součástí definice
jednotlivých příkazů, ale aby byla ve zprávě obsažena explicitně. Toto
rozhodnutí umožňuje například \textit{MTB-USB} modulu přeposílat zprávy mezi
\gls{mtbbus} a počítačem, aniž by \textit{MTB-USB} muselo dekódovat obsah
zpráv, případně mít velkou tabulku \textit{typ zprávy, délka}. Některé zprávy
navíc mohou mít variabilní délku ze své podstaty.

Maximální hodnota tohoto bytu je 121, což umožňuje zprávy skládat do bufferů o
délce 128 bytes. Maximální délka zprávy byla zvolena s ohledem na nevelké
kapacity \textit{SRAM} mikrokontrolérů a časovou efektivitu případné
retransmise.

Původní verze protokolu \gls{mtbbus} umožňovala zprávy o délce nejvýše 7 bytů.
Tato hodnota byla zásadně navýšena zejména proto, aby bylo možné novým
protokolem posílat firmware pro aktualizaci \gls{mtb} desek.

\item \textbf{Kód příkazu}

Libovolné číslo v rozsahu 0–255. Význam jednotlivých kódů definuje specifikace
protokolu.

\item \textbf{Data příkazu}

Až 120 bytů.

\item \textbf{Kontrolní součet} (2 byty)

Kontrolní součet se ve zprávě uvádí, aby bylo možné ověřit integritu přijímané
zprávy. Integrita zprávy může být narušena například elektromagnetickým rušením
nebo špatnými elektrickými vlastnostmi přenosového média.

Původní protokol \gls{mtbbus} zabezpečoval integritu zprávy pomocí jednoduchého
mechanismu \texttt{xor}, kdy na jednotlivé byty zprávy byl aplikován bitový xor
a výsledná hodnota byla připojena na konec zprávy. Tento způsob je vhodný pro
malé zprávy.

Nová verze protokolu již umožňuje delší zprávy a tak byl navržen robustnější
způsob zabezpečení zprávy. Sběrnice \gls{mtbbus} v4 používá pro zabezpečení
zprávy mechanismus \textit{CRC-16} \cite{crc16-modbus}. Tento mechanismus je
založen na dělení polynomů se zbytkem. Konkrétně se používá verze
\texttt{CRC-16-IBM}, inspirací autorovi byla průmyslová sběrnice \textit{ModBus}
\cite{modbus}, která tento způsob zabezpečení zpráv používá také.

\end{enumerate}

Příklad zprávy: \texttt{0x101 0x001 0x001 0x091 0x090}.

\begin{compactenum}
\item \texttt{0x101}: zpráva pro modul s adresou 1.
\item \texttt{0x001}: následuje 1 byte zprávy.
\item \texttt{0x001}: příkaz \texttt{0x01}.
\item \texttt{0x091}, \texttt{0x090}: kontrolní byty \textit{CRC-16}.
\end{compactenum}

Zpráva neobsahuje žádná data příkazu.

\subsubsection{Zprávy} \label{subsub:mtbbus-messages}

Kompletní specifikace zpráv a jejich odpovědí je dostupná na
\url{https://github.com/kmzbrnoI/mtbbus-protocol}, zde uvedeme stručný přehled.

Zprávy master $\rightarrow$ slave:

\begin{compactitem}
\item \textit{Module Inquiry}
\item \textit{Module Information Request}
\item \textit{Set Configuration}
\item \textit{Get Configuration}
\item \textit{Beacon}
\item \textit{Get Input}
\item \textit{Set Output}
\item \textit{Reset Outputs}
\item \textit{Change Address}
\item \textit{Change Speed}
\item \textit{Firmware Upgrade Request}
\item \textit{Firmware Write Flash}
\item \textit{Firmware Write Flash Status Request}
\item \textit{Module-specific Command}
\item \textit{Reboot}

\end{compactitem}

Zprávy slave $\rightarrow$ master:

\begin{compactitem}
\item \textit{Acknowledgement}
\item \textit{Error}
\item \textit{Module Information}
\item \textit{Module Configuration}
\item \textit{Input Changed}
\item \textit{Input State}
\item \textit{Output Set}
\item \textit{Firmware Write Flash Status}
\item \textit{Module-specific command}
\end{compactitem}



Rozeberme nyní, proč zprávy vypadají tak, jak vypadají.

Jak již bylo zmíněno, \textit{MTB-USB} deska průběžně skenuje aktivní
i neaktivní moduly (aktivní častěji) a posílá modulům \textit{Module Inquiry}.
Moduly odpovídají buď \textit{Acknowledgement} nebo \textit{Input Changed}.

Zpráva \textit{Set Output} slouží k nastavení stavu výstupů modulu, zpráv
\textit{Reset Outputs} slouží k navrácení výstupů do bezpečného stavu.
Tato zpráva je typicky odesílána \textit{MTB-USB} modulem při ukončení komunikace
s počítačem jako broadcast.

Počítač může zapisovat nebo vyčítat konfiguraci modulů pomocí zpráv
\textit{Get Configuration}, \textit{Set Configuration} a odpovědi
\textit{Configuration}.

Počítač může zapínat a vypínat indikační LED na \gls{mtb} modulech pomocí
zprávy \textit{Beacon}.

Modulům, které si ukládají adresu softwarově, může operátor z počítače změnit
adresu pomocí zprávy \textit{Change Address}.

Plánovaná změna rychlosti sběrnice se modulům ohlašuje pomocí zprávy
\textit{Change Speed}, která je typicky odesílána jako broadcast. Každý modul
si ukládá rychlost sběrnice do volatilní paměti a tuto rychlost použije při
zapnutí. Očekává se, že budoucí typy procesorů v modulech \gls{mtb} budou
schopny detekovat rychlost sběrnice automaticky. Aktuálně není na sběrnici
\gls{mtbbus} implementován žádný mechanismus pro automatickou detekci rychlosti.
V situaci, kdy je například do sběrnice doplněn nový modul, který je ale z
výroby nastaven na jinou rychlost, než aktuálně sběrnice používá, se očekává,
že operátor přenastaví rychlost sběrnice na rychlost nového modulu, odešle
novému modulu příkaz na změnu rychlosti a vrátí rychlost sběrnice zpět.
Tento mechanismus je přímočarý a lze provést za plného chodu sběrnice.

Vysvětlení procesoru aktualizace firmwaru modulů se budeme věnovat v samostatné
kapitole.

Poslední zajímavou zprávou je \textit{Module-specific command}. Tato zpráva
umožňuje tunelovat pro modul libovolná data. Pokud typ modulu podporuje nějaké
speciální operace, může tento typ zprávy využít k řízení těchto operací.


\subsubsection{Dvouvrstvost protokolu}

Specifikace protokolu je tzv. \textit{dvouvrstvá}. Protokol specifikuje typy
zpráv, ale specifikace obsahu některých zpráv může být různá pro různé typy
modulů. Specifickými vlastnostmi modulů jsou:

\begin{enumerate}
\item Formát dat vstupů, formát dat výstupů

Protokol je navržený tak, aby moduly mohly mít libovolné vstupy i výstupy
i jejich počty. Obsah datových bytů zpráv \textit{Set Output}, \textit{Input
Changed} a \textit{Input State} je proto specifikovaný ve specifikaci modulů.

Například modul \gls{mtbuni} má 16 analogových a 16 digitálních výstupů.
Protože počet vstupů i výstupů je malý, posílá se v protokolu vždy celý stav
vstupů a výstupů. Protokol počítá s tím, že jiné moduly mohou mít vstupy a
výstupy například s větším rozsahem hodnot. V takovém případě protokol umožňuje
to, aby se neposílal celý stav vstupů a výstupů, ale jen relevantní část.
Například příkaz \textit{Set Output} může poslat jen ty výstupy, jejichž
stav je třeba změnit. To je také důvodem, proč existují 2 různé zprávy
\textit{Input Changed} (ve které se posílá jen stav změněných vstupů) a
\textit{Input State} (ve které se posílá stav všech vstupů).
\footnote{Všimněte si, že tento mechanismus umožňuje také provádět polling
vstupů místo událostního hlášení změn stavů vstupů. To je zcela záměrný prvek
protokolu. Tento prvek se využije například u budoucích modulů s analogovými
vstupy.}

\item Formát konfigurace

Konfigurace různých typů modulů se mohou zásadně lišit v počtu
konfigurovatelných hodnot.

Například modul \gls{mtbuni} ma 24 nebo 26 konfiguračních bytů
\ref{uni:proto}. Proto se celá konfigurace modulu posílá najednou.  Protokol
však umožňuje libovolný způsob zasílá konfigurace: počítač například u modulů s
větším množstvím konfigurace může poslat rozsah adres, jejichž hodnoty chce
znát. Nebo naopak poslat rozsah adres a hodnoty na těchto adresách, které
nastavuje.

\item Adresování a data paměti pro aktualizaci firmware

Různé typy modulů typicky obsahují různé procesory, které jinak adresují paměť.
Liší se například šířka slova nebo velikost paměti. Význam datových bytů
zpráv pracujících s aktualizací firmwaru procesoru je proto definován pro
konkrétní typy modulů.

\end{enumerate}

\subsubsection{Typické fungování sběrnice}

Následující kroky popisují typické fungování systému \gls{mtb} (srovnejte
s~postupem definovaným na konci \ref{sec:mtbbus}).

\begin{compactenum}
\item Je zapnut řídicí počítač kolejiště.
\item Je připojeno napájení \textit{MTB-USB} desky a všech \gls{mtb} modulů.
\item \gls{mtb} moduly z paměti načtou konfiguraci, aplikují bezpečný stav
výstupů, začnou poslouchat na sběrnici.
\item \textit{MTB-USB} deska detekuje připojené moduly a zapamatuje si je.
\item Počítačová aplikace se připojí k \textit{MTB-USB} desce.
\item Počítačová aplikace vyčte z \textit{MTB-USB} desky seznam aktivních modulů.
\item Aplikace vyčte typ připojených modulů, zkonfiguruje je.
\item Řídicí SW kolejiště používá sběrnici – čte vstupy, nastavuje výstupy.
\item Při odpojení řídicího SW kolejiště je \gls{mtb} modulům poslán příkaz
	k resetování stavu výstupů, čímž je zajištěno, že při odpojení nebo pádu
	řídicího softwaru se budou nacházet výstupy v definovaném stavu.
\end{compactenum}

Systém \gls{mtb} však nevyžaduje striktní pořadí těchto kroků. Napájení
komponent může být zapnuto v libovolném pořadí, systém dokáže detekovat výpadky
jednotlivých prvků a obnovovat je. Pro ilustraci uveďme několik příkladů situací,
které v systému \gls{mtb} mohou nastat.

\begin{itemize}
\item Jiné pořadí zapnutí prvků.

Pokud je \textit{MTB-USB} zapnut až po zapnutí počítače, počítač tento stav
korektně detekuje (chybí CDC USB zařízení) a po připojení \textit{MTB-USB} jej
korektně inicializuje.

Sken všech adres sběrnice \gls{mtbbus} (255 adres) probíhá neustále, takže
pokud jsou moduly sběrnice zapnuty až po zapnutí napájení \textit{MTB-USB} desky,
jsou i tak korektně detekovány.

\item Výpadek \gls{mtb} modulu.

Pokud \gls{mtb} modul neodpovídá na požadavky, \textit{MTB-USB} to detekuje, nahlásí
situaci do počítače a počítač se podle toho zachová.

Pokud \textit{MTB-USB} deska detekuje nový \gls{mtb} modul, opět o tom zpraví
počítač, ten provede vyčtení typu desky a zkonfiguruuje ji.

\item Výměna modulu za chodu sběrnice

Starý modul je při odpojení napájení nahlášen jako neaktivní, po výměně modulu
je aktivován nový modul.

\item Změna adresy modulu za chodu sběrnice

Původní adresa je prohlášena za neaktivní (neodpovídá na požadavky), nová adresa
je nalezena a zkonfigurována.

\end{itemize}


\subsubsection{Aktualizace firmwaru \gls{mtb} modulů}

Sběrnice \gls{mtbbus} je navržena tak, aby skrze ní mohl být aktualizován
firmware \gls{mtb} modulů. Při návrhu mechanismu aktualizace firmwaru bylo
zvažováno, jestli proces aktualizace firmwaru nevést úplně jiným protokolem,
než \gls{mtbbus}. \gls{mtbbus} Je dostatečně obecný a navíc je třeba, aby
při aktualizaci jednoho modulu ostatní moduly sběrnice aktualizační příkazy
ignorovaly a nerušily tak aktualizaci. Tyto argumenty vedly k implementaci
mechanismu aktualizace firmwaru přímo do sběrnice \gls{mtbbus}. Výhodou tohoto
řešení je, že zbylé moduly sběrnice mohou při aktualizaci firmwaru jednoho
modulu nadále komunikovat, aktualizace tedy může probíhat za plného provozu.

Protokol počítá s tím, že přepis flash paměti v \gls{mtb} deskách může
probíhat pouze z bootloaderu \gls{mtb} desky. Aktualizační protokol proto
podporuje reboot procesoru do bootloaderu. Procedůra aktualizace firmwaru
probíhá následovně \footnote{Procedůra je podrobně popsaná v dokumentaci protokolu
\gls{mtbbus} na
\url{https://github.com/kmzbrnoI/mtbbus-protocol/blob/master/workflows.md}.}.

\begin{compactenum}
\item Master deska vyšle \textit{Firmware Upgrade Request}. Tím informuje \gls{mtb}
	modul, že má rebootovat do bootloaderu.
\item \gls{mtb} deska odpoví \textit{Acknowldgement} a rebootuje do bootloaderu.
\item \gls{mtbusb} deska pošle \gls{mtb} desce žádost o její obecné informace.
\item \gls{mtb} deska odpoví obecnými informacemi, ve kterých je mj. zaznačeno,
	že je v bootloaderu.
\item \gls{mtbusb} deska vyšle příkaz \textit{Firmware Write Flash}, ve kterém pošle
	blok flash paměti k zapsání a její adresu. Slave deska tuto paměť zapisuje.
\item \gls{mtbusb} modul se ptá \gls{mtb} modulu příkazem \textit{Firmware Write
	Flash Status Request}, jestli již byl zápis dokončen. Jakmile je zápis bloku
	paměti dokončen, přesouvá se na krok (5).
\item Jakmile je zapsána celá paměť, master deska pošle příkaz \textit{Reboot},
	\gls{mtb} modul provede kontrolu konzistence paměti a zapne se do hlavního
	programu.
\end{compactenum}

Jak bylo zmíněno v posledním kroku, je doporučeno (a nový firmware
\gls{mtbuni} modulů toto doporučení dodržuje), aby při každém náběhu
procesoru modulu proběhla kontrola konzistence paměti. Kontrola konzistence
paměti je implementována pomocí uloženého kontrolního součtu (\textit{CRC-16})
na konkrétních adresách paměti. Protokol \gls{mtbbus} implementuje atribut,
kterým může \gls{mtb} modul oznámit, že sice poslouchá, ale je v bootloaderu
a odmítá nabootovat, protože nesedí kontrolní součet paměti programu.

Aktualizace bootloaderu přes sběrnici \gls{mtbbus} není možná, protože vlivem
výpadku napájení při aktualizaci programu by mohlo dojít k situaci, kdy se modul
stane nepoužitelným. Bootloader by měl být malý a odzkoušený kus firmwaru, který
se nahraje jednou a nikdy ho nebude třeba přehrávat. Přehrání bootloaderu by
vyžadovalo ruční obcházení modulů s programátorem.
