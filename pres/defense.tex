\documentclass[aspectratio=169]{beamer}

\usepackage[czech]{babel}
\usepackage[utf8]{inputenc}
\usepackage[T1]{fontenc}
\usepackage{csquotes}
%\usepackage{biblatex}
%\addbibresource{defense.bib}
\usepackage{booktabs}
\usetheme[
  workplace=fi,
]{MU}

\title[MTB v4]{Návrh a implementace nového protokolu sběrnice MTBbus}
\subtitle[Obhajoba]{Obhajoba diplomové práce}
\author[J. Horáček]{Jan Horáček\texorpdfstring{\\}{, }horacekj@mail.muni.cz}
\institute[FI MU]{Fakulta informatiky Masarykovy univerzity}
\date{\today}
\subject{Návrh a implementace nového protokolu sběrnice MTBbus}
\keywords{mtb,mtbbus,stm32,avr,arm,pcb,rs485,protokol}
\begin{document}

\begin{frame}[plain]
\maketitle
\end{frame}

\section{Kontext}
\subsection{PLC}

%------------------------------------------------

\begin{frame}{Kontext: Programmable Logic Controller}
\begin{columns}
	\begin{column}{.6\textwidth}
		\begin{itemize}
		\item Základní jednotka průmyslové automatizace.
		\item Čte vstupy, nastavuje výstupy.
			\begin{itemize}
			\item Koncové snímače, motory, lasery apod.
		\end{itemize}
		\item Komunikuje s dalšími PLC obvody, s počítačem.
		\item RTOS, robustní.
		\item Řízení dopravních křižovatek, výrobních linek, elektráren apod.
		\item V této práci: nemá vlastní inteligenci.
		\begin{itemize}
		\item \textbf{PLC tvoří rozhraní mezi hardwarem a počítačem.}
		\end{itemize}
		\end{itemize}
	\end{column}
	\begin{column}{.4\textwidth}
		\begin{figure}
		\includegraphics[width=\columnwidth]{data/plc.jpg}
		\caption{Ukázka PLC modulů.}
		\end{figure}
	\end{column}
\end{columns}
\end{frame}

%------------------------------------------------

\subsection{Řízeni modelových kolejišť}

\begin{frame}{Kontext: PLC v řízení modelového kolejiště}
\begin{figure}
\includegraphics[width=0.8\columnwidth]{data/kolejisteMosilana.jpg}
\end{figure}
\end{frame}

%------------------------------------------------

\begin{frame}{Kontext: PLC v řízení modelového kolejiště}
\begin{columns}
	\begin{column}{.3\textwidth}
		\begin{itemize}
		\item 13 stanic.
		\item 186 výhybek.
		\item 306 úseků.
		\item 233 návěstidel.
		\item 13 přejezdů.
		\item 100 rozpojovačů.
		\item \textbf{70 MTB modulů.}
		\end{itemize}
	\end{column}
	\begin{column}{.65\textwidth}
		\begin{figure}
		\includegraphics[width=\columnwidth]{data/sk.png}
		\caption{Ukázka řídicího rozhraní kolejiště.}
		\end{figure}
	\end{column}
\end{columns}
\end{frame}

%------------------------------------------------

\begin{frame}{Systém MTB a jeho současné nasazení}
\begin{columns}
	\begin{column}{.6\textwidth}
		\begin{itemize}
		\item Vlastní systém vyvinutý okolo roku 2000.
		\item 1 kolejiště = 1 sběrnice MTBbus.
		\begin{itemize}
			\item 1 MTB-USB modul, více \textit{MTB modulů}.
		\end{itemize}
		\item \textit{Master-slave} topologie.
		\item Různé MTB moduly:
		\begin{enumerate}
			\item MTB-UNI, MTB-UNIm,
			\item MTB-TTL,
			\item MTB-REG,
			\item MTB-POT.
		\end{enumerate}
		\item Používá se pouze pro řízení \textit{příslušenství}, nikoliv
		pro řízení jízdy.
		\end{itemize}
	\end{column}
	\begin{column}{.4\textwidth}
		\begin{figure}
		\includegraphics[width=\columnwidth]{data/mtb-topology.pdf}
		\caption{Topologie systému MTB v2.}
		\end{figure}
	\end{column}
\end{columns}
\end{frame}

%------------------------------------------------

\begin{frame}{Problémy systému MTB (v KMŽ Brno I)}
\begin{enumerate}
\item Chybí výrobní data.
\item Licenční spory.
\item Zastaralý hardware.
\begin{itemize}
\item Nemožnost aktualizovat software.
\item Některé součástky se nevyrábí.
\end{itemize}
\end{enumerate}
\pause
\begin{alertblock}{Důsledky}
Nemožnost rozšiřovat kolejiště, omezená udržitelnost, nemožnost rozvoje systému.
\end{alertblock}
\pause
\begin{alertblock}{Závěr}
Je třeba přejít na nový systém.
\end{alertblock}
\end{frame}

%------------------------------------------------

\section{Návrh MTB v4}
\subsection{Požadavky}

\begin{frame}{Požadavky na MTB v4}
\begin{enumerate}
\item MTB-UNI – umožnit kmitavé výstupy.
\item MTB-UNI – umožnit S-COM výstupy na všech výstupních pinech.
\item Univerzální návrh pro různé typy modulů (i do budoucna).
\item Vyřešit zpětnou kompatibilitu se stávajícími moduly.
\item Přístup skrze \textit{hJOP RCS API}.
\item Více možných řídicích SW v počítači.
\item Hot-swap MTB modulů, detekce nefunkčních.
\item Identifikační LED.
\item Aktualizace FW MTB modulů po MTBbus.
\item Využít komponenty s dlouhodobou dostupností.
\item Opensource \& openhardware.
\end{enumerate}
\end{frame}

%------------------------------------------------

\subsection{Návrh}

\begin{frame}{Návrh MTB v4}
\begin{columns}
	\begin{column}{.3\textwidth}
		V rámci práce vznikl:
		\begin{enumerate}
		\item nový protokol MTBbus,
		\item MTB-USB v4,
		\item MTB-UNI v4,
		\item MTB-2-AVR,
		\item IRdet,
		\item MTB Daemon,
		\item hJOP MTB RCS Network Library.
		\end{enumerate}
	\end{column}
	\begin{column}{.4\textwidth}
		\begin{figure}
		\includegraphics[width=\columnwidth]{data/new-topology.pdf}
		\caption{Topologie systému MTB v4.}
		\end{figure}
	\end{column}
	\begin{column}{.3\textwidth}
		\begin{figure}
		\includegraphics[width=\columnwidth]{data/mtb-topology.pdf}
		\caption{Topologie systému MTB v2.}
		\end{figure}
	\end{column}
\end{columns}
\end{frame}

%------------------------------------------------

\subsection{MTBbus}

\begin{frame}{MTBbus v4}
\begin{columns}
	\begin{column}{.5\textwidth}
		\begin{itemize}
		\item Hardware sběrnice zachován: RS485.
		\item Protokol navrhnut od základů znovu.
		\begin{itemize}
			\item Rychlost zachována: $38\ 400$ – $115\ 200$ Bd
			\item Zachováno využití devitibitové komunikace.
			\item Zachován princip \textit{master-slave}.
		\end{itemize}
		\item Modul musí vždy odpovědět na každou zprávu.
		\end{itemize}
	\end{column}
	\pause
	\begin{column}{.5\textwidth}\texttt{\footnotesize
> Module 1 Inquiry \\
< Acknowledgement  \# Modul 1 žije, \\
  žádná data k odeslání \\
> Module 2 Inquiry \\
\# Timeout – modul 2 nežije \\
> Module 3 Inquiry \\
< State of inputs changed, \\
  new state: 0b00001010 0b11110000 \\
> Module 6 Inquiry \\
< Acknowledgement \\
> Set Outputs of module 1 to\\
  0b00101010 0b01000101 \\
< Outputs Set to 0b00101010 0b01000101 \\
> Module 10 Inquiry \\
< Acknowledgement \\
...}
	\end{column}
\end{columns}
\end{frame}

%------------------------------------------------

\begin{frame}{MTBbus v4}{Zpráva}
Každá \textit{zpráva} se skládá z devítibitových slov.
\begin{enumerate}
\item Adresa modulu (jen ve směru master $\rightarrow$ slave).
\item Délka zprávy.
\item Kód zprávy.
\item Data zprávy (až 120 bytů).
\item Kontrolní součet CRC-16 (2 byty).
\end{enumerate}

\begin{exampleblock}{Příklad zprávy: \texttt{0x105 0x001 0x001 0x0D0 0x051}}
\begin{enumerate}
\item \texttt{0x105}: zpráva pro modul s~adresou 5.
\item \texttt{0x001}: následuje 1 byte zprávy.
\item \texttt{0x001}: kód zprávy \texttt{0x01}.
\item \texttt{0x0D0}, \texttt{0x051}: kontrolní součet \textit{CRC-16}.
\end{enumerate}
\end{exampleblock}
\end{frame}

%------------------------------------------------

\begin{frame}{MTBbus v4}{Typy zpráv, dvouvrstvost zpráv}
\footnotesize
\begin{columns}
	\begin{column}{.45\textwidth}
		\texttt{0x01} \textit{Module Inquiry} \\
		\texttt{0x02} \textit{Module Information Request} \\
		\texttt{0x03} \textit{Set Configuration} \\
		\texttt{0x04} \textit{Get Configuration} \\
		\texttt{0x05} \textit{Beacon} \\
		\texttt{0x10} \textit{Get Input} \\
		\texttt{0x11} \textit{Set Output} \\
		\texttt{0x12} \textit{Reset Outputs} \\
		\texttt{0x20} \textit{Change Address} \\
		\texttt{0xE0} \textit{Change Speed} \\
		\texttt{0xF0} \textit{Firmware Upgrade Request} \\
		\texttt{0xF1} \textit{Firmware Write Flash} \\
		\texttt{0xF2} \textit{Firmware Write Flash Status Request} \\
		\texttt{0xFE} \textit{Module-specific Command} \\
		\texttt{0xFF} \textit{Reboot}
	\end{column}
	\begin{column}{.45\textwidth}
		\texttt{0x01} \textit{Acknowledgement} \\
		\texttt{0x02} \textit{Error} \\
		\texttt{0x03} \textit{Module Information} \\
		\texttt{0x04} \textit{Module Configuration} \\
		\texttt{0x10} \textit{Input Changed} \\
		\texttt{0x11} \textit{Input State} \\
		\texttt{0x12} \textit{Output Set} \\
		\texttt{0xF2} \textit{Firmware Write Flash Status} \\
		\texttt{0xFE} \textit{Module-specific Command}
	\end{column}
\end{columns}
\end{frame}

%------------------------------------------------

\section{Nový hardware}
\subsection{MTB-USB v4}

\begin{frame}{MTB-USB v4}
\begin{columns}
	\begin{column}{.4\textwidth}
		\begin{figure}
		\includegraphics[width=\columnwidth]{data/usb-all.jpg}
		\end{figure}
	\end{column}
	\begin{column}{.6\textwidth}
		\begin{figure}
		\includegraphics[width=\columnwidth]{data/usb-inside.jpg}
		\end{figure}
	\end{column}
\end{columns}
\end{frame}

%------------------------------------------------

\begin{frame}{MTB-USB v4}
\begin{enumerate}
\item Přeposílá data mezi MTBbus a počítačem.
\item Záměrně neobsahuje složitou logiku.
\item Vykonává časově kritické operace sběrnice MTBbus.
\item Provádí \textit{polling} modulů.
\item Udržuje seznam aktivních modulů.
\item Provádí retransmise zpráv do MTBbus při nedoručení.
\end{enumerate}
\end{frame}

%------------------------------------------------

\begin{frame}{Komunikační protokol s počítačem}
\begin{itemize}
\item Nejdůležitější příkaz: \textit{Forward Packet}.
\item Počítač může vyžádat seznam aktivních modulů.
\item MTB-USB posílá počítači asynchronně události na sběrnici.
\item Detailní popis protokolu v textu práce.
\end{itemize}

\begin{alertblock}{Zpráva}
Uvozena speciální sekvencí bytů \texttt{0x2A} \texttt{0x42}.
\end{alertblock}
\end{frame}

%------------------------------------------------

\begin{frame}{Hardware}
\begin{itemize}
\item USB a MTBbus část desky galvanicky oddělené.
\begin{itemize}
\item Napájení USB části z USB-C.
\item Napájení MTBbus části: přes měnič nebo externí napájení.
\end{itemize}
\item Procesor \texttt{STM32F103}.
\begin{itemize}
\item Přímá podpora USB.
\item V USB části DPS.
\end{itemize}
\item Budič RS485 \texttt{ADM2483}.
\item Automatické osazování na \textit{JLCPCB}.
\item KiCad.
\end{itemize}
\end{frame}

%------------------------------------------------

\begin{frame}{Firmware}
\begin{itemize}
\item C
\item STM32 HAL
\item DMA
\end{itemize}
\end{frame}

%------------------------------------------------

\subsection{MTB-UNI v4}

\begin{frame}{MTB-UNI v4}
\begin{columns}
	\begin{column}{.5\textwidth}
		\begin{itemize}
		\item 16 digitálních vstupů.
		\item 16 digitálních výstupů.
		\item Napájení 7–17 V DC.
		\item Adresování pomocí jumperů.
		\item Procesor ATmega128.
		\begin{itemize}
			\item Hlavní program a bootloader.
		\end{itemize}
		\end{itemize}
	\end{column}
	\begin{column}{.5\textwidth}
		\begin{figure}
		\includegraphics[width=\columnwidth]{data/uni-v40-screw-all.jpg}
		\caption{Ukázka modulu MTB-UNI v4.}
		\end{figure}
	\end{column}
\end{columns}
\end{frame}

%------------------------------------------------

\subsection{MTB-2-AVR}

\begin{frame}{MTB-2-AVR}
\begin{columns}
	\begin{column}{.5\textwidth}
		\begin{itemize}
		\item Použije se místo procesoru v původních MTB modulech.
		\item Univerzální deska pro MTB-UNI, MTB-UNIm a MTB-TTL.
		\item Upgrade sběrnice na v4 nevyžaduje nákladnou výměnu modulů.
		\item Procesor ATmega328p.
		\begin{itemize}
			\item Hlavní program a bootloader.
		\end{itemize}
		\end{itemize}
	\end{column}
	\begin{column}{.5\textwidth}
		\begin{figure}
		\includegraphics[width=\columnwidth]{data/uni-2-upgrade-all.jpg}
		\caption{Nástavná deska MTB-2-AVR v modulu MTB-UNI v2.}
		\end{figure}
	\end{column}
\end{columns}
\end{frame}

%------------------------------------------------

\subsection{IRdet}

\begin{frame}{IRdet}
\begin{columns}
	\begin{column}{.7\textwidth}
		\begin{itemize}
		\item Nahrazuje přímou podporu IR čidel v původních MTB-UNI.
		\item Není připojena na MTBbus.
		\item 8 IR čidel.
		\item 8 digitálních opticky oddělených výstupů.
		\item Univerzální.
		\end{itemize}
	\end{column}
	\begin{column}{.3\textwidth}
		\begin{figure}
		\includegraphics[width=\columnwidth]{data/irdet-front.jpg}
		\caption{Deska IRdet.}
		\end{figure}
	\end{column}
\end{columns}
\end{frame}

%------------------------------------------------

\section{Nový software}
\subsection{MTB Daemon}

\begin{frame}{MTB Daemon}
\begin{columns}
	\begin{column}{.5\textwidth}
		\begin{itemize}
			\item Umožňuje připojení více řídicích SW k MTB.
			\item Udržuje konfiguraci modulů.
			\item Poskytuje \textit{hezké} API.
		\end{itemize}
		Implementace:
		\begin{itemize}
			\item JSON TCP Server.
			\item USB CDC.
			\item C++.
			\item Qt.
			\item Multiplatformní.
		\end{itemize}
	\end{column}
	\begin{column}{.5\textwidth}
		\begin{figure}
		\includegraphics[width=\columnwidth]{data/new-topology.pdf}
		\caption{Topologie systému MTB v4.}
		\end{figure}
	\end{column}
\end{columns}
\end{frame}

%------------------------------------------------

\subsection{hJOP MTB Network Library}

\begin{frame}{hJOP MTB Network Library}
\begin{columns}
	\begin{column}{.5\textwidth}
		\begin{itemize}
			\item Integruje MTB v4 do aktuálně nasazeného SW \textit{hJOP} pro
			řízení kolejišť v KMŽ Brno I.
		\end{itemize}
		Implementace:
		\begin{itemize}
			\item C++.
			\item Qt.
			\item Dll.
			\item Windows.
		\end{itemize}
	\end{column}
	\begin{column}{.5\textwidth}
		\begin{figure}
		\includegraphics[width=\columnwidth]{data/new-topology.pdf}
		\caption{Topologie systému MTB v4.}
		\end{figure}
	\end{column}
\end{columns}
\end{frame}

%------------------------------------------------

\section{Závěr}

\begin{frame}{Závěr}
\begin{columns}
	\begin{column}{.5\textwidth}
		V rámci práce vznikl:
		\begin{enumerate}
		\item nový protokol MTBbus,
		\item MTB-USB v4,
		\item MTB-UNI v4,
		\item MTB-2-AVR,
		\item IRdet,
		\item MTB Daemon,
		\item hJOP MTB RCS Network Library.
		\end{enumerate}
	\end{column}
	\begin{column}{.5\textwidth}
		\begin{figure}
		\includegraphics[width=\columnwidth]{data/new-topology.pdf}
		\caption{Topologie systému MTB v4.}
		\end{figure}
	\end{column}
\end{columns}
\end{frame}

%------------------------------------------------

\begin{frame}{Nasazení}
\begin{itemize}
\item Květen 2021: menší stanice o 4 MTB modulech.
\item Červen 2021: jedno klubovní kolejiště o 20 MTB modulech.
\item Výhled červenec 2021: největší klubovní kolejiště o 70 MTB modulech.
\end{itemize}
\end{frame}

%------------------------------------------------

\begin{frame}{Přínos}
\begin{enumerate}
	\item Vznikl otevřený a spolehlivý systém pro řízení příslušenství modelových
		kolejišť, který může využít kdokoliv.
	\item Rozšíření bezpečného řízení modelové železnice.
	\item Zpřístupnění řízení kolejiště začínajícím programátorům.
\end{enumerate}

\begin{alertblock}{Přínos pro KMŽ Brno I}
\begin{enumerate}
\item Umožněno zprovoznění řízení silniční a tramvajové dopravy.
\item Umožněno řízení osvětlení.
\item Umožněno nasazení pultů obsluhy.
\item Umožněno vytvoření nových zesilovačů a RailCom modulů s cílem nasazení
	na lokomotivní depo. Otevřena cesta k vyšší spolehlivosti kolejiště.
\item Umožněna optimalizace výstupů MTB-UNI modulů.
\item Umožněna snazší diagnostika.
\end{enumerate}
\end{alertblock}
\end{frame}

\begin{frame}{Možná rozšíření}
\begin{enumerate}
\item Další typy MTB modulů.
\item Podpora MTB v komerčních SW pro řízení kolejišť.
\item Vyšší rychlosti sběrnice.
\item Automatická detekce rychlosti sběrnice.
\end{enumerate}
\end{frame}

%------------------------------------------------

%\section{\bibname}
%\begin{frame}[t, allowframebreaks]{\bibname}
%\printbibliography[heading=none]
%\end{frame}

%------------------------------------------------

\end{document}
