\documentclass[aspectratio=169]{beamer}

\usepackage[czech]{babel}
\usepackage[utf8]{inputenc}
\usepackage[T1]{fontenc}
\usepackage{csquotes}
\usepackage{soul}
\usepackage{booktabs}
\usetheme[
  workplace=fi,
]{MU}

\title[MTB v4]{Návrh a implementace nového protokolu sběrnice MTBbus}
\subtitle[Obhajoba]{Otázky oponenta}
\author[J. Horáček]{Jan Horáček\texorpdfstring{\\}{, }horacekj@mail.muni.cz}
\institute[FI MU]{Fakulta informatiky Masarykovy univerzity}
\date{\today}
\subject{Návrh a implementace nového protokolu sběrnice MTBbus}
\keywords{mtb,mtbbus,stm32,avr,arm,pcb,rs485,protokol}
\begin{document}

%------------------------------------------------

\section{Otázky oponenta}
\begin{frame}{Otázky oponenta}
\begin{alertblock}{Otázka I}
Bylo zachováno použití sběrnice RS-485 s tvrzením, že \uv{nepřináší žádné zjevné
nevýhody} (strana 39). Za předpokladu dodržování pravidel pro RS-485, se stejnou
kabeláží nebo s minimálními úpravami je možné použít rozhraní CAN.
Pokud porovnáte vlastnosti RS-485 a CAN, považujete ono tvrzení za pravdivé?
\end{alertblock}
\end{frame}

%------------------------------------------------

\begin{frame}{Otázky oponenta}

\begin{alertblock}{Otázka II}
Jakým způsobem se systém zachová, pokud je odpověď z modulu MTB-UNI se změnou
stavu vstupu odvysílána, není ale z důvodu krátkého zarušení modulem MTB-USB
přijmuta?
\end{alertblock}
\pause

\begin{columns}
	\begin{column}{.45\textwidth}
	\begin{exampleblock}{Korektní chování}
	\texttt{\footnotesize
> Module 10 Inquiry, C=0 \\
< State of inputs changed \\
... \\
> Module 10 Inquiry, C=1 \\
< Acknowledgement \\
... \\
> Module 10 Inquiry, C=0 \\
< Acknowledgement \\
...}
	\end{exampleblock}
	\end{column}
	\pause
	\begin{column}{.45\textwidth}
	\begin{exampleblock}{Zarušení}
	\texttt{\footnotesize
> Module 10 Inquiry, C=0 \\
< \st{State of inputs changed} \\
> Module 10 Inquiry, C=0 \\
< State of inputs changed \\
... \\
> Module 10 Inquiry, C=1 \\
< Acknowledgement \\
...}
	\end{exampleblock}
	\end{column}
\end{columns}
\end{frame}

%------------------------------------------------

\end{document}
