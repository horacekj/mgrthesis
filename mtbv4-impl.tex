\section{Implementace}

\subsection{\gls{mtbusb} v4}

\begin{figure}[ht]
%\includegraphics[width=0.7\textwidth]{data/mtb-topology.pdf}
\caption{Prototyp modulu MTB-USB.}
\label{fig:mtbusb-prototype}
\end{figure}

\gls{mtbusb} modul vzniknuvší v rámci této práce byl navržen kompletně nový,
inspirace starým \gls{mtbusb} modulem je minimální. Vznikly hardware, firmware
a popisy komunikačních protokolů.

Hlavním úkolem \gls{mtbusb} modulu je přeposílat data mezi sběrnici \gls{mtbbus}
a počítačem. \gls{mtbusb} modul provádí časově kritické operace sběrnice
 – například počítání timeoutu odpovědi na zprávu od \gls{mtb} modulů; pravidelné
dotazování \gls{mtb} modulů. Počítač pak informuje formou událostí.

\subsubsection{Komunikační protokol s počítačem}

Před návrhem samotné implementace je třeba navrhnout komunikační protokol
s počítačem. Tento komunikační protokol musí být přirozeně jiný, než nový
komunikační protokol sběrnice \gls{mtbbus}, který jsme popsali v předchozí
kapitole, protože zahrnuje jiné aktéry a funguje na jiné hardwarové platformě.

Komunikační protokol \gls{mtbusb} desky a počítače je navržen tak, aby byl do
velké míry nezávislý na protokolu sběrnice \gls{mtbbus}. Zásadním příkazem
je příkaz na přeposlání zprávy mezi \gls{usb} a \gls{mtbbus}. Tento příkaz umožňuje
počítačovému programu přeposlat libovolnou zprávu pro libovolný \gls{mtb} modul
(a naopak), aniž by \gls{mtbusb} deska musela znát sémantiku příkazů sběrnice
\gls{mtbbus}. Protokol sběrnice \gls{mtbbus} byl vytvářen přesně s tímto
cílem.

Na \gls{mtbusb} desku můžeme tedy pohlížet v zásadě jako na tenkého
přeposílatele mezi dvěma různými sběrnicemi – tzv. \textit{gateway}.

Plnohodnotná specifikace protokolu mezi počítačem a \gls{mtbusb} deskou je
k disopzici na \url{https://github.com/kmzbrnoI/mtbbus-protocol/tree/master/pc}.
Popišme nyní stručně návrh protokolu.

Mezi počítačem a \gls{mtbusb} deskou se komunikuje po virtuálním sériovém portu
(tzv. \textit{\gls{cdc}}) tunelovaným skrze \gls{usb} rozhraní. Toto řešení bylo vybráno,
protože je prakticky standardem pro připojením speciálních periferií k počítači.
Zvažován byl také například \textit{HID}. Samotná sběrnice \gls{usb}
nepodoporuje pojem zprávy tak, jak bychom ho vyžadovali\footnote{Při užití třídy \gls{cdc}
se data posílají nejčastěji každou milisekundu a to v bloku o nejvýše 64 bytech.
Počítačová aplikace díky bufferování operačního systému není schopna tyto bloky
spolehlivě rozlišit.}. Protože USB podporuje pouze 8bitový sériový port, začátek
zprávy je třeba označit jiným způsobem. v navrhovaném protokolu je začátek zprávy
označen speciální sekvencí dvou bytů \texttt{0x2A 0x42}. Tato sekvence se sice
ve zprávě může objevit, ale pravděpodobnost, že dojde k tolika chybám, aby
tato sekvence uprostřed zprávy byla považovaná za začátek zprávy, je
malá.\footnote{Uvědomme si, že rozdělování dat do zpráv na straně přijímače
se nemusí řídit jen detekcím magické sekvence bytů. Pokud zprávy nechodí moc
často, lze po delší době nepřijímání dat (třeba jednotky milisekund) vyprázdnit
vstupní buffer a očekávat, že další příchozivší data budou novou zprávou. Odesílací
strana nesmí odesílat jednotlivé části zprávy s velkou prodlevou, to je ale
rozumný požadavek.}

Celková struktura zprávy vypadá následovně (op jednotlivých bytech):

\begin{compactenum}
\item \texttt{0x2A},
\item \texttt{0x42},
\item počet následujících bytů,
\item kód příkazu,
\item data (až 122 bytů).
\end{compactenum}

Struktura je podobná struktuře zprávy protokolu \gls{mtbbus}, viz
\ref{subsub:mtbbus-proto-strucure}. Pozorný čtenář si přesto všimne například
chybějícího kontrolní součtu. Ten chybí, protože integritu zprávy řeší přímo
sběrnice \gls{usb} a tak ji není třeba řešit znovu. Upozorňujeme, že
\textit{kód příkazu} není kód příkazu sběrnice \gls{mtbbus}, ale kód příkazu
protokolu PC – \gls{mtbusb}.

Zprávy od počítače pro \gls{mtbusb} modul jsou:

\begin{itemize}
\item \textbf{Forward packet to \gls{mtbbus}}

V \textit{datech} příkazu následuje příkaz pro \gls{mtb} modul a adresa modulu,
kterému má být příkaz poslán.

\item \textbf{MTB-USB Information Request}

\item \textbf{Change Speed}

Požadavek na změnu komunikační rychlosti \gls{mtbusb} modulu. Změnu rychlosti
jednotlivých \gls{mtb} modulů je třeba provést předchozím příkazem, typicky
broadcastem všem \gls{mtb} modulům.

\item \textbf{Active modules request}

Odopovědí na tento příkaz je seznam aktivních adres \gls{mtb} modulů.
\end{itemize}

Zprávy od \gls{mtbusb} desky pro počítač:

\begin{itemize}
\item \textbf{Acknowledgement}
\item \textbf{Error}
\item \textbf{Packet from \gls{mtbbus}}

Zpráva je odeslána počítači při odpovědi \gls{mtb} modulu na příkaz počítače
nebo na pravidelný sken \gls{mtb} modulů. Z pohledu počítače tak přichází jak
odpovědi na příkazy pro \gls{mtb} modul, které poslal počítač, tak asynchronní
události – například změna stavu vstupů.

\item \textbf{\gls{mtbusb} Information}

\item \textbf{Active modules list}

\item \textbf{New module discovered event}

\item \textbf{Module failed event}

\end{itemize}

\gls{mtbusb} deska si udržuje seznam aktivních adres sběrnice. \textit{Polling}
modulů probíhá v iteracích. V každé iteraci jsou osloveny všechny aktivní
moduly a 10 neaktivních modulů. Tím je zaručeno, že aktivní moduly jsou
skenovány často a zároveň jsou detekovány nové moduly.

Pokud \gls{mtb} modul neodpoví na \textit{Module Inquiry}
(\ref{subsub:mtbbus-messages}), počítači je odeslána zpráva \textit{Module
failed event}. Modul je považován za ztracený, jakmile neodpoví na výzvu
ve třech po sobě jdoucích iteracích. Zpráva \textit{Module failed event}
je tedy při výpadku modulu odeslána třikrát. Zpráva obsahuje počet zbývajících
pokusů. Zpráva počítači o neodpovězení modulu je odeslána při každém neodpovězení,
nikoliv až při finálním označení modulu za neaktivní, aby bylo možné z počítače
monitorovat chod sběrnice. Občasné neodpovídání modulů na výzvy je dobrým
indikátorem problémů se sběrnicí.

Vlastností \gls{mtbbus} protokolu je, že každý \gls{mtb} modul musí vždy
odpovědět na každou zprávu, kterou dostane\footnote{Výjimkou jsou pouze
\textit{broadcast} zprávy.}. Protože \gls{mtbbus} je potenciálně nespolehlivé
médium, na kterém integritu příkazů kontrolujeme vlastními mechanismy, provádí
modul \gls{mtbusb} retransmisi zpráv pro \gls{mtb} moduly v případě, že na zprávy
nepřijde žádná odpověď, a to až třikrát. Proto součástí zprávy \textit{Packet
from \gls{mtbbus}} je také počítadlo, které říká, na kolikátý pokus byla tato
odpověď přijata (u asynchronních událostí 0). Toto číslo bylo do protokolu
vloženo opět se záměrem, aby bylo možné v počítači monitorovat chod sběrnice.

Na dalších příkazech protokolu není vcelku nic zajímavého, pokud je čtenář chce
prostudovat, je mu k dispozici plná specifikace protokolu na
\url{https://github.com/kmzbrnoI/mtbbus-protocol/tree/master/pc}.

\subsubsection{Hardware}

Srdcem desky je procesor \texttt{STM21F103}. Jedná se o moderní mikrokontrolér
rodiny \textit{ARM}, který autor zvolil z několika důvodů.

\begin{compactenum}
\item Procesor \texttt{STM32F103} má hardwarovou podporu \gls{usb}.
\item Procesory \texttt{STM32} nabízí velké velikosti pamětí a výpočetní výkon.
\item Procesory \texttt{STM32} se stávají standardy ve vestavěných systémech.
\item Procesor \texttt{STM32F103} je jako jeden z mála \texttt{STM} součástí
	\textit{basic} součástek na \url{https://jlcpcb.com/}.
\item Autor se chtěl naučit pracovat s novou architekturou procesorů.
\end{compactenum}

Zastavme se krátce u některých bodů.

Hardwarová podpora \gls{usb} umožňuje výrazně vyšší flexibilitu komunikace
práce s počítačem, než použití historicky zaužívaných převodních obvodů mezi
\gls{usb} a sériovou linkou procesoru. V programu procesoru je tak například
možné definovat, že procesor má více \gls{cdc} linek – druhá se hodí například
pro ladění programu. Procesor může používat libovolnou třídu \gls{usb}. Celé
řešení se tak stává mnohem lépe upravitelné pouze změnou softwaru, což je
zásadně méně pracné, než změna hardwaru.

Všechny desky plošných spojů navrhnuté v rámci této práce, jsou navrženy tak,
aby se daly automaticky osazovat na \url{https://jlcpcb.com/}\footnote{Aktuálně
lze osazovat pouze jednu stranu desky a pouze \textit{SMD} součástky.}. Tato
firma nabízí velice levné automatické osazování malých sérií desek, což výrazně
zjednodušuje nasazení nových desek plošných spojů. JLCPCB rozlišuje tzv.
\textit{basic} a \textit{extended} součástky k osazení, přičemž za
\textit{basic} se neplatí režijní poplatek při osazování. \textit{basic}
součástky jsou ty, které se používají skutečně často (typicky diskrétní
součástky – rezistory, kondenzátory apod.) a které se budou vyrábět i za
desítky let.

Schéma a výkres desky plošných spojů byly vytvořeny v nástroji \textit{KiCad},
který autor této práce používal poprvé. Jedním z hlavních přínosů celé této
práce pro něj je, že se naučil pracovat s novými nástroji. Schéma a výkres
desky plošných spojů jsou k dispozici
online\footnote{\url{https://github.com/kmzbrnoI/mtb-usb-4-pcb}}, schéma je
přiloženo jako příloha této práce (\ref{}).

Popišme nyní stručné schéma. Na první pohled je schéma rozděleno na 2 části,
které jsou na desce plošných spojů (\textit{\gls{dps}}) galvanicky oddělené –
\gls{usb} část (vlevo) a \gls{mtbbus} část (vpravo). \gls{usb} část obsahuje
procesor a je napájena přímo z \gls{usb}. \gls{mtbbus} část obsahuje rozhraní
sběrnice \gls{mtbbus} a je napájeno buď z externího zdroje (ze stejného jako
zbytek sběrnice \gls{mtbbus}) nebo přes galvanicky oddělený měnič \texttt{PS1}.
Při osazení desky se osazením nebo neosazením součástek zvolí, která varianta
se bude používat.

Hlavním prvkem \gls{usb} části schématu je již zmíněný procesor
\texttt{STM32F03} (vlevo nahoře). K procesoru jsou připojena diagnostická
rozhraní (\texttt{J2}, \texttt{J3}). Celá \gls{usb} část na napájena z \gls{usb}
portu, přičemž je využito moderního konektoru \gls{usb}-C.

Rozhraní \gls{usb} a \gls{mtbbus} části schématu tvoří galvanicky oddělený
driver sběrnice RS485 typu \texttt{ADM2483}. Jedná se o osvědčený driver,
který zvládá proudy sběrnice až do $200~mA$ \cite{adm2483-datasheet} a je tedy
vhodný pro komunikaci s větším počtem \gls{mtb} modulů.

Specialitou \gls{mtbusb} modulu je měřící obvod \texttt{INA219}, který měří
napětí a proud do sběrnicové části obvodu \texttt{ADM2483} a posílá naměřenou
hodnotu do procesoru (skrze galvanické oddělení sběrnice \textit{I2C}). Počítač
je tak schopen detekovat nestandardní chování sběrnice, například zkrat.
\footnote{Pro tento hardware zatím neí podpora v komunikačním protokolu
s~počítačem a ve firmwaru. Hardware je odzkoušený. Podporu je v~plánu doplnit
v~další verzi.}

Při návrhu desky plošných spojů bylo hlavním problémem do jaké krabičky desku
vložit. \gls{mtbusb} modul totiž typicky není pevnou součástí kolejiště, je
umístěn vedle kolejiště u řídicího serveru. Vyvstal tak požadavek modul zavřít
do krabičky, ideálně s možností uchycení na zeď. Po rešerši byla zvolena krabička
firmy \textit{Digikeijs}, jejíž zásadní výhodou je to, že pro vyvedení konektorů
a indikačních LED není třeba do krabičky frézovat. Navíc lze na krabičku velice
elegantně nalepit potisk, který vysvětluje použití jednotlivých konektorů
a~význam LED. Celé řešení je zobrazeno na úvodním obrázku
\ref{fig:mtbusb-prototype}.


\subsubsection{Firmware}

Retransmise kdo

\subsubsection{Protokoly}

\subsection{\gls{mtbuni} v4}


\subsection{MTB-2-AVR}


\subsection{IRdet}


\subsection{MTB daemon}


\subsection{hJOP MTB Network RCS knihovna}
