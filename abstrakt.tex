Se zaváděním počítačového řízení železniční dopravy vznikaly systémy umožňující
centrálnímu řídicímu počítači interakci s~venkovními prvky zabezpečovacího
zařízení – výhybkami, návěstidly apod. Tato práce se zaměřuje na návrh
a~implementaci sběrnice, která takovou interakci umožní, ovšem na kolejištích
modelových. Práce navrhuje a~implementuje nový protokol sběrnice pro řízení
modelových kolejišť \textit{MTBbus}. Je popsáno, proč je současný systém řízení
kolejiště nedostatečný, jsou formulovány požadavky na nový systém, tento systém
je implementován. V~rámci práce vznikl detailní návrh nového protokolu, nové
hardwarové moduly pro řízení kolejiště, modul pro komunikaci s~počítačem,
obslužný počítačový software a~knihovna integrující nový hardwarový systém se
současným řídicím softwarem kolejiště.  Nový řídicí systém byl otestován
a~nasazen na skutečná kolejiště, čímž umožnil jejich další rozšiřování
a~zprovoznění nových způsobů řízení dopravy. Vznikl otevřený a~robustní systém
s~výhledem dlouhodobé udržitelnosti, který svými vlastnostmi převyšuje mnohé
současné komerční systémy řízení modelových kolejišť.  Vzniknuvší systém je
obecný, takže jej lze použít i~v~mnoha jiných aplikacích.
